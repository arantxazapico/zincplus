\documentclass[11pt,letterpaper,usenames,dvipsnames]{article}

%%%%%%%%%%%%%%%%%%%%%%%%%%%%%%%%%%%%%%%%%%%%%%%%%%%%%%%%%%%%%%%%%%%%%%%%%%%%%%%%
% PACKAGES
\usepackage{amsmath,amssymb,amsthm}
\usepackage{fullpage}
\usepackage{multirow}
\usepackage{bbm,soul}
\usepackage[usenames,dvipsnames]{xcolor}
\usepackage{lipsum}
\usepackage{makecell}
\usepackage{bm}
\usepackage{braket}
\usepackage{adjustbox,mdframed}
\usepackage[operators,sets,primitives,asymptotics,lambda,keys,ff]{cryptocode}
\usepackage{enumitem}
\usepackage{geometry}
\usepackage{booktabs}
\usepackage[font=footnotesize,labelfont=bf]{caption}
\usepackage{mleftright}
\usepackage{wrapfig}
\usepackage{setspace}
\usepackage{xspace}
\usepackage{algorithm,algorithmic}
\floatname{algorithm}{Protocol}
%\crefname{algorithm}{Protocol}{Protocols}
%\Crefname{algorithm}{Protocol}{Protocols}
\renewcommand{\algorithmiccomment}[1]{// #1}
\renewcommand{\algorithmicrequire}{\textbf{Input:}}
\renewcommand{\algorithmicensure}{\textbf{Output:}}
\usepackage{authblk}

\usepackage{framed}
\usepackage{comment}
\usepackage{placeins} % in the preamble

\usepackage{float}
\usepackage{graphicx}
\graphicspath{ {./figures/} }
\usepackage[T1]{fontenc}
\usepackage{titlesec}
\usepackage{stmaryrd}

\usepackage[backend=biber,giveninits,style=alphabetic,maxalphanames=6,maxnames=6,backref]{biblatex}
\addbibresource{references.bib} 
%\addbibresource{cryptobib/abbrev3.bib} 
%\addbibresource{cryptobib/crypto_crossref.bib}

%%%%%%%%%%%%%%%%%%%%%%%%%%%%%%%%%%%%%%%%%%%%%%%%%%%%%%%%%%%%%%%%%%%%%%%%%%%%%%%%
\newcommand{\plaintitle}{Zinc{+}}
\newcommand{\stylizedtitle}{\textsf{Zinc{+}}}
\PassOptionsToPackage{hyphens}{url}\usepackage{hyperref}
\hypersetup{
    colorlinks=true,
    linkcolor=CadetBlue,
    breaklinks=true,
    citecolor=Periwinkle,      
    urlcolor=Emerald,
    pdftitle={\plaintitle},
    pdfpagemode=FullScreen,
}
\DefineBibliographyStrings{english}{%
  backrefpage = {cited on p.},% originally "cited on page"
  backrefpages = {cited on pp.},% originally "cited on pages"
}
% Remove starred sections from toc
\DeclareRobustCommand{\SkipTocEntry}[5]{} 
\usepackage[capitalize, nameinlink]{cleveref}
\usepackage{tikz}

%%%%%%%%%%%%%%%%%%%%%%%%%%%%%%%%%%%%%%%%%%%%%%%%%%%%%%%%%%%%%%%%%%%%%%%%%%%%%%%%

\newcommand{\PsiV}[1]{{\color{red}Psi: #1}}
\newcommand{\Albert}[1]{{\color{purple}Albert: #1}}
\newcommand{\Aru}[1]{{\color{blue}Aru: #1}}




\newcommand{\gp}{\mathsf{gp}}
\newcommand{\commit}{\mathsf{Commit}}
\newcommand{\open}{\mathsf{Open}}
\newcommand{\evaluation}{\mathsf{Evaluation}}
\newcommand{\testingP}{\mathsf{Testing Phase}}
\newcommand{\evaluationP}{\mathsf{Evaluation Phase}}
\newcommand{\com}{\mathsf{com}}


\newcommand{\code}{\mathcal{C}}

\newcommand{\polyspace}{(\QQ[X])[\vec Y]}


\newcommand{\encode}{\mathsf{Enc}}


\newcommand{\size}{2^\mu}
\newcommand{\matrixU}{\mathbf{\hat{u}}}
\newcommand{\matrixV}{\mathbf{V}}
\newcommand{\vectorUi}{\mathbf{\hat{u}_i}}
\newcommand{\oracleUi}{\llbracket\mathbf{\hat{u}}_i\rrbracket}

%\newcommand{cmd}{def}

%%%%%%%%%%%%%%%%%%%%%%%%%%%%%%%%%%%%%%%%%%%%%%%%%%%%%%%%%%%%%%%%%%%%%%%%%%%%%%%%
\title{\stylizedtitle}

\newcommand{\FormatAuthor}[3]{%
\begin{tabular}{c}
#1 \\ {\small\texttt{#2}} \\ {\small #3}
\end{tabular}
}

\author{
	Albert Garreta, Psi Vesely, Arantxa Zapico
}
\date{\today}

\newcommand{\albert}[1]{\textcolor{teal}{Albert: {#1}}}
\newcommand{\alberturgent}[1]{\textcolor{red}{Albert: {#1}}}

%%%%%%%%%%%%%%%%%%%%%%%%%%%%%%%%%%%%%%%%%%%%%%%%%%%%%%%%%%%%%%%%%%%%%%%%%%%%%%%%
\begin{document}
\maketitle

\begin{abstract}
\end{abstract}

\setcounter{tocdepth}{3}
\begin{spacing}{0.8}
{\footnotesize \tableofcontents}
\end{spacing}
%%%%%%%%%%%%%%%%%%%%%%%%%%%%%%%%%%%%%%%%%%%%%%%%%%%%%%%%%%%%%%%%%%%%%%%%%%%%%%%%
%%%%%%%%%%%%%%%%%%%%%%%%%%%%%%%%%%%%%%%%%%%%%%%%%%%%%%%%%%%%%%%%%%%%%%%%%%%%%%%%
%%%%%%%%%%%%%%%%%%%%%%%%%%%%%%%%%%%%%%%%%%%%%%%%%%%%%%%%%%%%%%%%%%%%%%%%%%%%%%%%
\newcommand{\cR}{\mathcal{R}}

\section{Introduction}


\section{Preliminaries}

        

\subsection{Rings and rings of polynomials} Throughout the paper we extensively work with multivariate polynomials whose coefficients are polynomials themselves. Formally, given a ring $\cR$ and variables $\YY=(Y_1,\ldots, Y_\mu)$, we let $\cR[\YY]$ denote the ring of multivariate polynomials on variables $\YY$, with coefficients inn $\cR$. For us, typically $\cR$ will be of the form $\ZZ[X]$ or $\QQ[X]$ (or even $\local{q}[X]$ for some prime $q$). In that case, $\cR[\YY]$ contains all polynomials on variables $\YY$ whose coefficients are polynomials from $\ZZ[X]$ (or $\QQ[X]$, or $\local{q}[X]$). 

The ring $(\ZZ[X])[\YY]$ is isomorphic to the ring $\ZZ[X, \YY]$ of all polynomials with integer coefficients on variables $(X,\YY)$  (and similarly for $(\QQ[X])[\YY]$ and $(\local{q}[X])[\YY]$). However, in this work we do not use this isomorphic representation of our polynomials, and insist on viewing elements from $(\ZZ[X])[\YY]$ as polynomials on the variables $\YY$ with coefficients in $\ZZ[X]$. We denote such a polynomial by $$f(X;\YY),$$
highlighting the presence of the variable $X$, even though strictly speaking we look at $f$ as a polynomial on the variables $\YY$. The reason for doing so is that sometimes we will want to evaluate $X$ at a specific value $x$ \albert{I think we are not doing that anymore because `` evaluation at $x$'' is equivalent to membership to the ideal $(X-x)$} \alberturgent{Decide if keeping the $X$ soon, otherwise it will be too difficult to remove --or put a macro for it}, obtaining a polynomial $f(x;\YY)$ in $\ZZ[\YY]$, and sometimes we will need to evaluate the variables $\YY$ at a specific values $\yy$, and we will look at the result as  a polynomial $f(X,\yy)$ in $\ZZ[X]$. 

In general, we denote vectors of elements with lowercase boldface letters, e.g.\ $\vv,\uu,$ etc. Given a ring $\cD$ (typically $\ZZ, \QQ$, or $\local{q}$), we denote vectors of elements from $(\cD[X])[\YY]$ as $\vv(X;\YY), \uu(X;\YY)$, etc.


For every ring $\cR$  we fix a publicly known representation of its elements as strings of bits. By $\cR_{\leq B}$ we denote the subset of $\cR$ formed by all elements whose bitstring representation contains at most $B$ bits. In \cref{s: bistring_reps} we describe the representation we use for the rings $\ZZ,\QQ, \ZZ[X],$ and $\QQ[X]$ \albert{And $\local{q},\local{q}[X]$?}.

\alberturgent{In case of polynomial rings $\cR[X]$, do we need to specify a degree bound? Technically not, but having a single size parameter could allow for wiggle room of degrees, making some descriptions and analysis a bit more tricky. Also Zip+ is designed having in mind that there's a fixed degree bound. We could have $B$ be a finite subset of $\cR$ instead of a bit bound}

\subsection{Commutative algebra}
All rings considered in this paper are associative, commutative, and have a multiplicative identity. Let $\cR$ be a ring. A subset $I$ of $\cR$ is an \emph{ideal} of $\cR$ if for all $a\in I$ and $b\in \cR$, one has $a\cdot b \in I$, and $I$ is additively closed, i.e.\ $a_1 + a_2 \in I$ for all $a_1, a_2\in I$. The ideal generated by a set of elements $S\subseteq \cR$ is the set $\{\sum_{s\in S} a_s \cdot s \mid (a_s)_{s\in S}\in \cR^{|S|}\}$. An ideal is \emph{prime} if for all $a,b\in \cR$ such that $a\cdot b \in I$, one has that either $a$ or $b$ belongs to $I$. 

Given a ring $\cR$ and an ideal $I$ of $\cR$, the \emph{quotient ring} of $\cR$ by $I$ is the set  $\cR/I =\{a+I \mid a\in \cR\}$. The set $\cR/I$ is itself a ring under the natural addition and multiplication operations.  The map $\pi:\cR\to \cR/I$ sending each $a\in \cR$ to $a+I$ is a ring homomorphism which we call the \emph{natural projection of $\cR$ onto $\cR/I$.}

A ring $\cR$ is an \emph{integral domain} if for all nonzero $a,b\in \cR$ we have $a\cdot b\neq 0$. Every integral domain $\cR$ is a subring of a field (namely, the field of fractions of $\cR$). The following result is well-known:
\begin{remark}\label{r: domain_prime}
Let $\cR$ be a ring and let $I$  be an ideal of $\cR$. Then $\cR/I$ is an integral domain if and only if $I$ is a prime ideal.
\end{remark}

Given a ring $\cR$ and a tuple of variables $\YY$, the set $\cR[\YY]$ formed by all polynomials on variables $\YY$ and coefficients in $\cR$ forms a ring. Given two rings $\cR_1$ and $\cR_2$, any ring homomorphism $f:\cR_1 \to \cR_2$  extends naturally to a ring homomorphism $\cR_1[\YY] \to \cR_2[\YY]$ by applying $f$ coeffcient-wise to the coefficients of the polynomials in $\cR_1[\YY]$ (with respect to any basis like the monomial basis). 

Let $X$ be a single variable and $\cR$ a ring. A polynomial $f(X)\in \cR[X]$ is \emph{irreducible} if it is nonzero and whenever $f(X)=a(X)\cdot b(X)$ for some $a(X), b(X)\in \cR[X]$, we have that either $a(X)$ or $b(X)$ is constant (i.e.\ it has degree $0$). The following result is well-known:
\begin{remark}\label{r: prime_ideals_of_RX}
\albert{Prove?} An ideal $I$ of a ring $\cR[X]$ is prime if and only if it is generated by a non- irreducible polynomial of $\cR[X]$ or by a prime element of $\cR$.
\end{remark}

The following version of Schwartz-Zippel lemma applies to polynomials of arbitrary degree. We state and prove it only for multilinear polynomials because this is the only scenario where we will need it. Its generalization is straightforward.
\begin{lemma}[Schwartz-Zippel lemma for (possibly infinite) integral domains]\label{l: SZ_for_integral_domains}
%
Let $\dom$ be an integral domain, possibly infinite, and let $S$ be a finite subset of $\dom$. Let $\YY$ be a tuple of variables and let $f(\YY)\in D^\multilin[\YY]$ be a non-zero polynomial. Then the number of elements $\rr\in S^{|\YY|}$ such that $f(\rr)=0$ is at most $S^{|\YY|-1}$. 
%
\end{lemma}
\begin{proof}
Let $\YY=(Y_1,\ldots, Y_\mu)$. We proceed by induction on $\mu$. If $\mu=1$ then $f(\YY) = a Y_1 + b$. If $a=0, b\neq 0$ then $f$ has no roots. If $a\neq 0$ then any root $r$ satisfies $a\cdot r + b = 0$. In particular, $r = -b\cdot a^{-1}$ where $a^{-1}$ is the inverse of $a$ in the field of fractions of $\dom$. If $\mu > 1$, write  $f(\YY) = a(Y_1,\ldots, Y_{\mu-1}) \cdot Y_\mu +  b(Y_1,\ldots, Y_{\mu-1})$ for some multilinear polynomials $a,b\in D^\multilin[Y_1,\ldots, Y_\mu]$. If $a$ is the zero polynomial, then the result follows by induction. Assume $a$ is not the zero polynomial, and let  $\rr=(r_1,\ldots, r_{\mu}) \in  S^{\mu}$ be such that $f(\rr)=0$. Then either $a(r_1,\ldots, r_{\mu-1})=0$, or $r_\mu = -b(r_1,\ldots, r_{\mu-1})\cdot a(r_1,\ldots, r_{\mu-1})^{-1}$ where, as before, the inverse is taken in the field of fractions of $\dom$.  Hence the set of roots of $f$  is contained in the set $\{(r_1,\ldots,r_\mu)=(\rr',r_\mu) \in S^\mu \mid a(\rr')= 0, \text{ or } r_\mu = -b(\rr')\cdot a(\rr')^{-1}\}$. By induction, this latter set has size at most $|S|^{\mu-1}$, and the lemma follows. 
%
\end{proof}


\subsection{Bitstring representation of ring elements}\label{s: bistring_reps}
\albert{To do.}

\subsection{Reductions of knowledge}

\section{Algebraic constraints over $\QQ[X]$ modulo prime ideals}

In this section we define the type of constraints we are interested in working on. As we argued in \cref{?}, and as we see later in \cref{s: example_sha}, these constraints are highly expressive, and can  express essentially all computations of interest with almost no arithmetization overhead.

\newcommand{\evalconstraints}{\text{\emph{Eval}}}
\newcommand{\mainQpolyring}{(\QQ[X]_{\leq B})}
\newcommand{\mainQpolyringmultilin}{(\QQ[X]_{\leq B})^{\multilin}}

%\begin{definition}  Let $I$ be a prime ideal of $\cQ[X]$. A \emph{system of algebraic equations modulo l $I$} is a set of polynomials with coefficients in $\QQ[X]$ (possibly multivariate and of arbitrary degree), and $I$ is a prime ideal of the ring $\QQ[X]$. We assume $|\cQ|$ is a power of two.
%\end{definition}


%and index the entries in $\cQ$ with elements from $\{0,1\}^{\log|\cQ|}$.
%

\newcommand{\megaQ}{\mathfrak{Q}}
\renewcommand{\ff}{\mathbf{f}}
\newcommand{\cS}{\mathcal{S}}

Let $\gp = (k, m, n, \mu, B)$ be global parameters, where $k,m,n,B$ are size parameters whose meaning will become apparent soon. 

Let $I_1,\ldots, I_t$ be a tuple of prime ideals of $\QQ[X]$. For each $i\in [t]$, let $\cQ_t$ be a set  of \emph{constraint polynomials} with coefficients in $\QQ[X]$ (possibly multivariate and of arbitrary degree), with $|\cQ_t|$ a power of two.  The polynomials in $\cQ_i$ will be used to place constraints on instances and witnesses.
Denote 
%
$$
\megaQ = \{(\cQ_1, I_1), \ldots, (\cQ_t, I_t)\}.
$$
%
Let $\cS=\{S_1, \ldots, S_k\}$ be a collection of finite subsets of $\QQ[X]$.
%Let $\evalconstraints$ be a set of triples $(i, j, a)$ where $i,j\in [k]$ and $a \in \QQ$. 
%
 
 An algebraic relation for $(\gp,\megaQ,\cS)$ is a set $\REL_{\gp, \cQ,\cS}$ of triples $(\idx, \inp;\wit)$ with the following properties:
\begin{itemize}
	\item \textbf{Index.} The \emph{index} $\idx$ contains $n$ oracles \albert{maybe $n$ should be the size of the witness, it's more standard} $\oracle{g_1(X;\YY)}, \ldots, \oracle{g_n(X;\YY)}$ to multilinear polynomials from $\mainQpolyring^\multilin[\YY]$, where $\YY=(Y_1,\ldots, Y_\mu) $ is a tuple of variables.
	 \item \textbf{Constraint polynomials.} For each $i\in [t]$, $\cQ_t$ is a set of polynomials %$$\cQ_t=\{Q_{t,\bb}\}_{\bb\in \{0,1\}^{\log|\cQ_t|}}$$
     with coefficients in $\QQ[X]$, each on $(n+ k)\cdot 2^\mu + m$ variables.
	\item \textbf{Witness} The \emph{witness} $\wit$ is consists of $k$ multilinear polynomials $f_{1}(X;\YY),\ldots, f_{k}(X;\YY))$ from $\mainQpolyring^{\multilin}[\YY]$. %We denote these vectors by $\ff_1(X;\YY), \ldots, \ff_r(X;\YY)$ and their entries as $\ff_{i}(X;\YY)=(f_{i1}(X;\YY),\ldots, f_{ik}(X;\YY))$. %and the multilinear polynomials $g_1,\ldots, g_n\in \R_B^\multilin[\XX]$.
	\item \textbf{Public instance.} The public instance $\inp=(\zz(X), \oracle{f_1(X;\YY)}, \ldots, \oracle{f_k(X;\YY))})$, where $\zz\in \mainQpolyring^m$.
	\item \textbf{Algebraic constraints modulo prime idelas.} For each  $i\in [t]$  and $Q\in \cQ_t$, we have that $Q$ belongs to $I_i$  when evaluated on the values $$\vv_{\idx, \inp,\wit}(X)=(\yy(X), (g_1(X;\yy),\ldots, g_n(X;\yy), f_1(X,\yy),\ldots, f_k(X;\yy))_{\yy\in \BB^\mu}),$$
    i.e.
    $$
    Q(\vv_{\idx, \inp,\wit}(X)) \in I_t.
    $$
    \item \textbf{Membership constraints.} For each $i\in [k]$ be have that $f_i(X;\yy) \in S_i$ for all $\yy\in \{0,1\}^\mu$.
    %\item For each  polynomial $Q \in \cQ$, we have that $Q$ vanishes  when evaluated on the values $$\vv_{\idx, \inp,\wit}(X)=(\yy(X), (g_1(X;\yy),\ldots, g_n(X;\yy), f_1(X,\yy),\ldots, f_k(X;\yy))_{\yy\in \BB^\mu}),$$
    %i.e.
    %$$
    %Q(\vv_{\idx, \inp,\wit}(X)) =0.
    %$$
    %\item \albert{New thing wrt Zinc!} For each $(i,j,a)\in \evalconstraints$ we have
    %
    %$$
    %f_i(a; \yy) = f_j(a;\yy) \quad \text{for all } \yy\in \{0,1\}^\mu.
    %$$
    %\albert{Should we allow for X-evaluations of elements in the public instance?}
	\end{itemize}
%
    Formally, $\REL_{\gparams,\megaQ, \cS}$ has the following form:
    %
    \begin{equation*}
    \begin{aligned}
    \REL_{\gparams,\megaQ,\cS} = \left\{ (\idx, \inp ; \wit) \left| \ \begin{aligned}
    &\textbf{Index-instance-witness}\\
    &\quad \idx = ( \oracle{g_1(X;\YY)},\ldots, \oracle{g_n(X;\YY)}), \ \YY=(Y_1,\ldots, Y_\mu), \\
            &\quad(g_1(X;\YY),\dots,g_n(X;\YY))\in \left(\mainQpolyringmultilin[\YY]\right)^{n},\\
    &\quad \inp =(\yy(X), \oracle{f_1(X;\YY)}, \ldots, \oracle{f_k(X;\YY)}),\\ &\quad \yy(X)\in \mainQpolyring^{m},\\    
        &\quad \wit = (f_1(X;\YY), \ldots, f_k(X;\YY)) \in \left(\mainQpolyringmultilin[\YY]\right)^{k},\\
        &\textbf{Algebraic constraints mod ideal}\\
        &\quad Q(\vv_{\idx, \inp,\wit}(X)) \in I_i  \text{ for all } i\in [t], \ Q\in \cQ_i,\\
        &\quad \vv_{\idx, \inp,\wit}(X) = \left(\begin{aligned}&\yy(X),\\ &\left(\begin{aligned}&g_1(X;\yy),\ldots, g_n(X;\yy),\\ &f_1(X,\yy),\ldots, f_k(X;\yy)\end{aligned}\right)_{\yy\in \BB^\mu}\\
     \end{aligned}\right),\\
     &\textbf{Membership constraints}\\
        &\quad f_i(X,\yy)\in S_i \text{ for all } i\in [k], \yy\in \{0,1\}^\mu
        %& f_i(a;\yy)=f_j(a;\yy)  \text{ for all } \yy\in \{0,1\}^\mu \text{ and } (i,j,a)\in \evalconstraints
    \end{aligned} \right.\right\}
    \end{aligned}
    \end{equation*}

    \begin{remark}[The membership constraints are algebraic constraints]
    The membership constraints, namely $f_i(X,\yy) \in S_i$ for all $i\in [k]$, $\yy\in \{0,1\}^\mu$, can be written as algebraic constraints of the form $Q(\vv_{\idx,\inp,\wit}(X))=0$ for a certain collection of polynomials $Q$. See \cref{??} for a proof. 
    
    Hence, the membership constraints could be removed from the definition of $\REL_{\gp, \megaQ, \cS}$,  without altering its expressiveness. Later, however, we will want to prove membership constraints with lookup-type techniques, while dealing with the algebraic constraints with other type of schemes such as Spartan \cite{spartan} inspired ones. This is the reason why membership constraints are specified in $\REL_{\gp,\megaQ, \cS}$ with their own separate syntax.
    \end{remark}

    \begin{remark}
        \albert{Emphasize that membership constraints are proved via a lookup ran over a random prime field $\FF$ and that the SOS decomposed witness only needs to be committed over $\FF$}
    \end{remark}

    \begin{definition}[Strictly algebraic relation $\REL_{\gp, \megaQ}$]\label{d: strictly_algebraic} We say that  $\megaQ$ is \emph{strictly algebraic} if $\megaQ$ contains only one pair $(\cQ, I)$ and $I$ is the zero ideal.  In that case,  the constraints $$Q(\vv_{\idx, \inp,\wit}(X))\in I_i, \quad \text{ for all } i\in [t],\ Q\in \cQ_i$$ from $\REL_{\gp,\megaQ}$ become $$Q(\vv_{\idx, \inp,\wit}(X))=0, \quad  \text{ for all }Q\in \cQ.$$
    \end{definition}
    
\subsection{Examples}\label{s: example_sha}

\subsubsection{Mixing the bitwise XOR operation and arithmetic modulo $2^n$}

\newcommand{\and}{\ \mathsf{AND}\ }
\newcommand{\xor}{\ \mathsf{XOR}\ }
\newcommand{\eq}{\mathsf{eq}}

As a simple example, consider the function 
%
\begin{align*}
f: [0, 2^{32}-1] &\to [0, 2^{32}-1]\\
u &\mapsto (u\cdot u \mod 2^{32}) \xor u
\end{align*}
%
where the operation $u\cdot u \mod 2^{32}$ here means squaring $u$ and taking as a result $u'$ the reduction of $u^2$ modulo $2^{32}$ in the range $[0,2^{32}-1]$. Then,  $u'\xor u$ means performing the bitwise XOR operation between $u'$ and $u$ when $u'$ and $u$ are represented as strings of $32$ bits. After the bitwise operation is performed, the resulting string of $32$ bits $u''$ is reinterpreted as an integer in the range $[0, 2^{32}-1]$. 

Say that a prover wants to show that it knows the result of computing $f^{2^{20}-1}(u_1)$ for some public integer $u\in [0, 2^{32}-1]$. Here $f^{i}(u_1)$ denotes the result of consecutively applying $f$ for $i$ times, starting with the input $u_1$. 


We proceed to describe an algebraic  constraint over $\QQ[X]$ modulo prime ideals, $\REL_{\gp, \megaQ,\cS}$, encoding the computation $f^{2^{20}-1}(w_1)$.

We begin by introducing the following notation: Denote the set of all polynomials of degree strictly less than $32$ with binary coefficients as $\{0,1\}^{< 32}[X]$. Consider the following bijection:
%
\begin{align*}
\phi: [0,2^{32}-1] &\to \{0,1\}^{< 32}[X]\\
v= \sum_{i=1}^{32} v_i\cdot 2^{i-1} &\mapsto \sum_{i=1}^{32} v_i \cdot X^{i-1},
\end{align*}
%
above, the elements $v_i$ all belong to $\{0,1\}$ and $v= \sum_{i=1}^{32} v_i\cdot 2^{i-1}$ corresponds to the base-2 representation of $v$.  Now, define the following elements and vectors  for all $i\in [ 2^{20}-1]$. We omit referring to $X$ even though many of these elements and vectors depend on $X$:
\begin{align*}
    u_i' &= (u_i \cdot u_i \mod 2^{32}) &&\in  [0, 2^{32}-1],\\
    u_{i+1} &=  f(u_i)=\phi^{-1}\left(\phi(u_i') \xor \phi(u_i)\right) &&\in \{0,1\}^{\leq 31}[X],\\
       % w_{i+1} &= f(w_i) = \phi^{-1}(\hat{w_{i+1}}) &&\in [0, 2^{32}-1],\\ 
     \uu&=(u_1,\ldots, u_{2^{20}}) &&\in [0,2^{32}-1]^{2^{20}},\\
     \uu' &= (u_1',\ldots, u_{2^{20}-1}', 0) &&\in [0,2^{32}-1]^{2^{20}},\\
          \phi(\uu) &= (\phi(u_1),\ldots, \phi(u_{2^{20}}))&& \in (\{0,1\}^{< 32}[X])^{2^{20}},\\
     \phi(\uu') &= (\phi(u_1'),\ldots, \phi(u_{2^{20}-1}'),0) &&\in (\{0,1\}^{< 32}[X])^{2^{20}}.%\\
     %\hat{\ww} &= (0,\hat{w_2}, \ldots, \hat{w_{2^{20}}}) &&\in (\{0,1\}^{< 32}[X])^{2^{20}}
\end{align*}
The $0$'s at the end of $\uu', \phi(\uu')$ can be chosen arbitrarily: our computation never accesses these entries of the vectors, and so these entries are not be subject to any constraint whatsoever.


We introduce $4$ multilinear polynomials on $20$ variables $\YY=(Y_1,\ldots, Y_{20})$, with coefficients in $\QQ[X]$. We denote them by $f_{\uu}(X; \YY), f_{\uu'}(X;\YY), f_{\phi(\uu)}(X;\YY), f_{\phi(\uu')}(X;\YY)$. These polynomials will be the witness of the index-instance-witness triple we will build. Intuitively, $f_{\uu}(X; \YY), f_{\uu'}(X;\YY), f_{\phi(\uu)}(X;\YY), f_{\phi(\uu')}(X;\YY)$ are supposed to be the multilinear extensions of the vectors $\uu, \uu', \phi(\uu)$, and $\phi(\uu')$, respectively. 

We start by imposing the following membership constraints: For all $\yy\in \{0,1\}^{20}$,
%
\begin{equation}\label{e: example_membership_constraints}
f_{\uu}(X;\yy), f_{\uu'}(X;\yy) \in [0,2^{32}-1], \quad f_{\phi(\uu)}(X;\yy), f_{\phi(\uu')}(X;\yy) \in \{0,1\}^{<32}[X].
\end{equation}
%
For ease of notation, in what follows we represent elements from $\{0,1\}^{20}$ as integers in the range $[2^{20}]$. 

Assuming the membership constraints \eqref{e: example_membership_constraints} hold, we have that  
%
$$
f_{\uu'}(X; i)= (f_{\uu}(X;i)\cdot f_{\uu}(X;i) \mod 2^{32}) \quad \Leftrightarrow  \quad f_{\uu'}(X; i)= f_{\uu}(X;i)\cdot f_{\uu}(X;i) + 2^{32} \mu_i(X)
$$
(the symbol $\Leftrightarrow$ should be read as ``if and only if'')
for some $\mu_i(X)\in \ZZ[X]$, for all $i \in [2^{20}-1]$. Above, the operations and equality on the right-hand side  are all happening in the ring $\QQ[X]$.   Note that if \eqref{e: example_membership_constraints} holds, then automatically $$\mu_i(X) \in [0, 2^{32}-1],$$
this being relevant because we will later need to commit include the values $\mu_i$ as part of the witness, which will need to be committed eventually and we'd like our witnesses to not blow up in size. We emphasize that the observation $\mu_i(X) \in [0, 2^{32}-1]$  does not need to be enforced as it follows  directly from the previous constraints.


Similarly, for all $i\in [2^{20}-1]$,
%
\begin{align*}
&f_{\phi(\uu)}(X; i+1)= f_{\phi(\uu')}(X;i)\xor f_{\phi(\uu)}(X;i)\\ &\quad \quad \quad \quad \quad \quad \quad \quad \quad \quad \Leftrightarrow \quad  f_{\phi(\uu)}(X; i+1) - (f_{\phi(\uu')}(X;i) + f_{\phi(\uu)}(X;i)) + 2\cdot \eta_i(X) \in I_{GF}
\end{align*}
\alberturgent{Actually for this operation we can just equate to $0$ since the membership constraints allow for $\eta_i(X)=0$ to be a valid choice (or is it forced to be $0$?) }for some $\eta_i \in \ZZ[X]$,  where $I_{GF}$ is a fixed ideal generated by  a polynomial of degree $32$ which is irreducible both in $\FF[X]$ and in $\FF_2[X]$ \albert{needs more detail, maybe in a separate section}. Similarly as above, note that if \eqref{e: example_membership_constraints} holds, then automatically $$\eta_i(X) \in \{0,1\}^{<31}[X].$$  Again, this membership does not need to be enforced, as it is a logical consequence of previous constraints.

Further, as before assuming that \eqref{e: example_membership_constraints} holds, we have that for all $i\in [2^{20}-1]$  \albert{turn into its own lemma}
%
$$
f_{\phi(\uu)}(X; i) = \phi(f_{\uu}(X; i)) \quad \Leftrightarrow\quad f_{\phi(\uu)}(X;i)-f_{\uu}(X;i) \in (X-2) 
$$
where $(X-2)$ denotes the ideal of $\QQ[X]$ generated by the polynomial $X-2$. And similarly,
$$
f_{\phi(\uu')}(X; i) = \phi(_1'(X; i)) \quad \Leftrightarrow\quad f_{\phi(\uu')}(X;i)-f_{\uu'}(X;i) \in (X-2). 
$$
%
% 
We now have all we need to put together a relation of the form $\REL_{\gp, \megaQ, \cS}$ which encodes the computation $f^{2^{20}-1}(u_0)$.

We set $\gp=(k=6,m=1,n=0,\mu=20, B)$ \albert{what $B$?}. The $k=6$ witness polynomials are $$\wit=(f_{\uu}(X;\YY), f_{\uu'}(X;\YY), f_{\phi(\uu)}(X;\YY),  f_{\phi(\uu')}(X;\YY), \mu(X;\YY), \eta(X;\YY)),$$ where $\mu(X;\YY), \eta(X;\YY)$ satisfy $\mu(X;i)=\mu_i(X)$ and $\eta(X;i)=\eta_i(X)$ for each $i\in \{0,1\}^{20}$. Then we set $\cS=(S_1, \ldots, S_6)$ with
%
$$S_1 = S_2 = [0,2^{32}-1],\ S_3=S_4= \{0,1\}^{< 32}[X],\ S_5=S_6=\mainQpolyringmultilin.$$
%
Effectively, these choice of sets places the membership constraints from \cref{e: example_membership_constraints}. The selection $S_5, S_6$ is done so that there is effectively no membership constraint placed on the elements  $\mu(X;i)$ and $\eta(X;i)$, other than the already ``native'' membership constraint to the set of polynomials from $\QQ[X]$ representable with at most $B$ bits.

We further have $\idx = \emptyset$ since we chose $n=0$. Now, we have $\yy(X) \in \mainQpolyring^m$ must consist of a single polynomial because we chose $m=1$. We let $\yy(X) = (w_0)$, i.e.\ $\yy(X)$ contains simply the initial public integer $w_0$ that our computation begins with. Then we have $$\inp = \left(\begin{aligned}&\yy(X)=(w_0),\\ &\oracle{f_{\uu}(X;\YY)}, \oracle{f_{\phi(\uu)}(X;\YY)}, \oracle{f_{\uu'}(X;\YY)}, \oracle{f_{\phi(\uu')}(X;\YY)}, \oracle{\mu(X;\YY)}, \oracle{\eta(X;\YY)}\end{aligned}\right).$$
%
Denote
%
$$
\vv_{\inp,\wit}(X)=(w_1, f_{\uu}(X;\yy), f_{\uu'}(X;\yy), f_{\phi(\uu)}(X;\yy), f_{\phi(\uu')}(X;\yy), \mu(X;\yy), \eta(X;\yy))_{\yy\in \{0,1\}^{20}}.
$$
Finally, we select the tuple $\megaQ$ as follows. We let $\megaQ=\{(\cQ_1, I_1), (\cQ_2, I_2), (\cQ_3, I_3), (\cQ_4, I_4), (\cQ_5, I_5)\}$,
where for all $i\in [2^{20}-1]$ (recall we identify $\{0,1\}^{20}$ and $[2^{20}]$)
%
\begin{align*}
    &I_1 = \{0\}, &&\cQ_1=(Q_{1i})_{i\in [2^{20}-1]},\\
    &I_2 = I_{GF}, &&\cQ_2=(Q_{2i})_{i\in [2^{20}-1]},\\
    &I_3 =(X-2), &&\cQ_3=(Q_{3i})_{i\in [2^{20}]},\\
      &I_4 =(X-2), &&\cQ_4=(Q_{4i})_{i\in [2^{20}-1]},
\end{align*}
with
\begin{align*}
    Q_{1i}(\vv_{\inp,\wit}(X))&=f_{\uu'}(X; i) - f_{\uu}(X;i)\cdot f_{\uu}(X;i) - 2^{32} \mu_i(X),&& \text{for all } i\in [2^{20}],\\
    Q_{2i}(\vv_{\inp,\wit}(X))&=f_{\phi(\uu)}(X; i+1) - (f_{\phi(\uu')}(X;i) + f_{\phi(\uu)}(X;i)) + 2\cdot \eta_i(X),&& \text{for all } i\in [2^{20}-1],\\
    Q_{3i}(\vv_{\inp,\wit}(X))&=f_{\phi(\uu)}(X;i)-f_{\uu}(X;i), && \text{for all } i\in [2^{20}],\\
    Q_{4i}(\vv_{\inp,\wit}(X))&=f_{\phi(\uu')}(X;i)-f_{\uu'}(X;i), && \text{for all } i\in [2^{20}-1].
\end{align*}
We also want to make sure that $f_{\uu}(X; 1)= u_1$. We add this extra constraints as a new tuple  $(\cQ_5, I_5)$ where $I_5=\{0\}$ and $\cQ_5=\{Q_5\}$ with $Q_5(\vv_{\inp,\wit})= f_{\uu}(X;1)-u_1$.

Finally, we saw that all our witness polynomials, when evaluated on $\{0,1\}^{20}$, either belong to the range $[0, 2^{32}-1]$, or to the set $\{0,1\}^{< 32}[X]$. Hence we set $B$ as:
%
$$
B=\text{maximum bit size of an element from } [0,2^{32}-1] \cup \{0,1\}^{< 32}[X],
$$
for our fixed bitstring representation of elements from $\QQ[X]$.

Putting everything together, we obtain the following relation for the computation $f^{2^{20}-1}(w_0)$ (again, recall we identify the sets $\{0,1\}^{20}$ and $[2^20]$):
   \begin{equation*}
    \begin{aligned}
    &\REL_{\gparams,\megaQ,\cS} =\\ &\left\{ \left(\begin{aligned}&\idx,\\ &\inp;\\ &\wit\end{aligned}\right) \left| \ \begin{aligned}
        &\textbf{Index-instance-witness}\\
    &\quad \idx = \emptyset, \ \YY=(Y_1,\ldots, Y_{20}), \inp =(u_1),\\    
        &\quad  \wit = \left(\begin{split}&f_{\uu}(X;\YY), f_{\uu'}(X;\YY),\\ &f_{\phi(\uu)}(X;\YY),  f_{\phi(\uu')}(X;\YY),\\ &\mu(X;\YY), \eta(X;\YY)\end{split}\right) \in \left(\mainQpolyringmultilin[\YY]\right)^{6},\\
        &\textbf{Algebraic constraints mod ideal}\\
        &\quad f_{\uu}(X;1)-u_1 = 0,\\
        &\quad f_{\uu'}(X; i) - f_{\uu}(X;i)\cdot f_{\uu}(X;i) - 2^{32} \mu_i(X) = 0 \text{ for all } i\in [2^{20}-1],\\
        &\quad f_{\phi(\uu)}(X; i+1) - (f_{\phi(\uu')}(X;i) + f_{\phi(\uu)}(X;i)) + 2\cdot \eta_i(X) \in I_{GF} \text{ for all } i\in [2^{20}-1],\\
        &\quad f_{\phi(\uu)}(X;i)-f_{\uu}(X;i) \in (X-2)\text{ for all } i\in [2^{20}],\\
        &\quad f_{\phi(\uu')}(X;i)-f_{\uu'}(X;i) \in (X-2)\text{ for all } i\in [2^{20}-1],\\
        &\textbf{Membership constraints}\\
        &\quad  f_{\uu}(X;i), f_{\uu'}(X;i) \in [0,2^{32}-1], \text{ for all } i \in \{0,1\}^{20},\\ 
        &\quad f_{\phi(\uu)}(X;i), f_{\phi(\uu')}(X;i) \in \{0,1\}^{<32}[X],  \text{ for all } i \in \{0,1\}^{20}
        %& f_i(a;\yy)=f_j(a;\yy)  \text{ for all } \yy\in \{0,1\}^\mu \text{ and } (i,j,a)\in \evalconstraints
    \end{aligned} \right.\right\}
    \end{aligned}
    \end{equation*}

    Overall, there are $6$ witness multilinear polynomials on $20$ variables, $1 +4\cdot 2^{20}$ algebraic constraints modulo ideals, and $4$ membership constraints for $4$ of the $6$ witness polynomials. 

    \albert{How would this compare to a constraint fully expressed over a binary field, or over M31? Ask someone? }
    
\subsubsection{Other bitwise operations: bitwise AND, bit rotation, etc.}

\subsubsection{The SHA-256 compression function}

The SHA-256 compression function is applied to bitstrings of length $512$. At all times, there are $8$ variables $H_0,\ldots, H_7$ that are periodically updated.  These variables are initialized as specific constants.




\subsubsection{SHA-256 hashing + RSA signature verification}

\subsubsection{Lattice-based operations}


\subsection{Reduction to strictly algebraic relations}

 In this section we provide a reduction of knowledge from  any relation of the form $\REL_{\gp, \megaQ}$ to a strictly algebraic relation $\REL_{\gp, \megaQ'}$ (cf.\ \cref{d: strictly_algebraic}), i.e.\ a relation where  $\megaQ'$ consists of just one pair $(Q', I)$ and $I$ is the zero  ideal. 


    \begin{lemma}
    Let $f(X;\YY) \in (\QQ[X])^{\multilin}[\YY]$ be a multilinear polynomial on variables $\YY= (Y_1,\ldots, Y_\mu)$ and coefficients in $\QQ[X]$. Let $i(X) \in \QQ[X]$ be a non-constant irreducible polynomial, and let $I$ be the ideal of $\QQ[X]$ generated by $I$. 
    
    Assume $f(X;\yy)\not\in I$ for some $\yy\in \QQ^\mu$. Let $S$ be a finite subset of $\QQ$. Then the number of elements $\rr\in S^\mu$ such that $f(X;\rr)\in I$ is a most $|S|^{\mu-1}$.
    \end{lemma}
    \begin{proof}
    Let $\dom= \QQ[X]/I$ and let  $\pi:\QQ[X] \to D$ be the natural projection of $\QQ[X]$ onto $\dom$. The ring $\dom$ is an integral domain by \cref{r: domain_prime}, and so $\dom$ is a subring of a field $\KK$. Let $\pi:(\QQ[X])[\YY]\to D[\YY]$ be the natural extension of $\pi$ for polynomials (cf.\ \cref{??}). Let $g(\YY) = \pi(f(X;\YY))$, i.e.\ $g(\YY)$ is the result of applying $\pi$ to the coefficients of $f$ (with respect to any basis, like the monomial basis). 
    
     If the polynomial $g(\YY) \in D[\YY]$ was the zero polynomial, then we would have $g(\yy) = 0$, which means that $\pi(f(X;\yy))\in I$, contradicting our assumptions. Hence $g(\YY)$ is not the zero polynomial. Moreover, $|\pi(S)|=|S|$, because whenever $\pi(s_1) = \pi(s_2)$ for some $s_1,s_2\in S$, we have $s_1 - s_2 \in  I$. But $S$ contains only constant polynomials, while all elements of $I$ are either zero or nonconstant polynomials (because $I$ is generated by a non-constant polynomial). Hence the only possibility is that $s_1=s_2$.  We conclude that $\pi$ is a bijection between $S$ and $\pi(S)$.
     
     Now, by Swartz-Zippel lemma for (possibly infinite) integral domains (cf.\ \cref{l: SZ_for_integral_domains}), the number of elements $\rr'\in \pi(S)^\mu$ such that $g(\rr')=0$ is at most $|\pi(S)|^{\mu-1}= |S|^{\mu-1}$. On the other hand, if $\rr\in S^\mu$ is such that $f(X;\rr)\in I$, then $\pi(f(X;\rr))=g(\pi(\rr))=0$ \albert{technically the first equality should be justified}, and so $\pi(\rr)$ is an element from $\pi(S)^{\mu}$ such that $g(\pi(\rr))=0$. Since $\pi$ is a bijection, the result follows.
    \end{proof}

     	\begin{algorithm}[H]
		\caption{A 
        reduction of knowledge from $\cR_{\gp, \cQ}$ to a strictly algebraic relation $\cR_{\gp, \cQ'}$. \label{a: red_to_strictly_algebraic}}
		\textbf{Input:} $\Prover$, $\verifier$ receive inputs $(\idx, \inp; \wit)\in \REL_{\gp, \cQ}$. %Write $\inp=(\xx, \oracle{\tilde{\ww}}, \oracle{\tilde{\uu}})$, with $(\ww, \uu)\in \ZZ_B^{2n-k-1}$,  $\xx\in \ZZ_B^{k}$.
		\begin{algorithmic}[1]
            \STATE $\verifier$ samples $\rr\in S^\mu$ uniformly at random, for a fixed finite subset of $\QQ$. 
            \STATE $\prover$ replies with an alement $a(X)\in \QQ[X]$ such that
            %
            $$
            \sum_{\bb\in \{0,1\}^{\log|\cQ|}} Q_b(\vv_{\idx,\inp,\wit}(X))\cdot \eq(\bb; \rr) = a(X)
            $$
            \STATE 
		\end{algorithmic}
	\end{algorithm}



%%%%%%%%%%%%%%%%%%%%%%%%%%%%%%%%%%%%%%%%%%%%%%%%%%%%%%%%%%%%%%%%%%%%%%%%%%%%%%%%
%%%%%%%%%%%%%%%%%%%%%%%%%%%%%%%%%%%%%%%%%%%%%%%%%%%%%%%%%%%%%%%%%%%%%%%%%%%%%%%%
%%%%%%%%%%%%%%%%%%%%%%%%%%%%%%%%%%%%%%%%%%%%%%%%%%%%%%%%%%%%%%%%%%%%%%%%%%%%%%%%

\section{Zinc+ PIOP}





\section{Zip+: A polynomial commitment scheme for $(\QQ[X])[\vec Y]$}
\subsection{The protocol}

\newpage

\begin{figure}[H]
	\begin{framed}
		\begin{description}
			\item[$\commit(\gp, f)$:] $f\in\polyspace$, $\vec Y\in\FF^{2\mu}$ 
				\begin{itemize}
					\item Compute matrix $V^f$ of size $\size\times\size$ and coefficients in $\QQ[X]$ that represents $f$ the following way:
						$$\mathbf V^f=(f_{b_1, b_2}(X))_{b_1, b_2 \in \{0,1\}^\mu}\in \QQ[X]^{\size\times\size}$$
					\item For each $i\in[\size]$, compute
					 $\vectorUi=\encode(f_{i,b_2}(X))_{b_2\in\{0,1\}^\mu}\in\QQ[X]^n$ and matrix 
					 $\matrixU=(\vectorUi)_{i\in [\size]}\in\QQ[X]^{\size\times n}$
					\item Output $\com=(\oracleUi)_{i\in[\size]}$, where $\oracleUi$ denote oracles to $\vectorUi$
				\end{itemize}
			\item[$\open(\gp, \com, f, \matrixU)$:] \ \
				\begin{itemize}
					\item Parse $(\vectorUi)_{i\in[\size]}\gets\matrixU$ and $(\oracleUi)_{i\in[\size]}\gets\com$
					\item Check that $\com$ consists of oracles to $\vectorUi$ and that $\vectorUi$ is $\delta$-close to $\encode_\code(\matrixV^f_i)$ for all $i$. \Aru{how}
					\item Reject if at any moment reading $\hat{u}$, $f$, or $\com$ some coefficient is not in $\QQ[X]$ or is larger than $poly()$
				\end{itemize}		
		\item[$\evaluation$:] \
		\item[$\testingP:$]
				\begin{enumerate}
					\item $V$ sends $r_1, \ldots, r_{\size}\in[0, q_0-1]$ and $\mathbf \alpha=(\alpha_0, \ldots, \alpha_{Bdeg})\in[0, q_0-1]^{Bdeg}$
					\item $P$ computes and outputs 
					$$\mathbf{v}=\sum_{i\in[\size]}r_i\matrixV^f_i(\mathbf{\alpha})\in \QQ^{\size}$$
					\item $V$ randomly chooses $J\subset[n]$ with $|J|=\Theta(\delta)$ and for each $j\in J$
						\begin{itemize}
							\item If $\mathbf v_j$ is not an integer or $|\mathbf v_j|>...$, $V$ rejects
							\item Queries $\hat{u}_{1,j}(X), \ldots, \hat{u}_{\size, j}(X)\in \QQ[X]$
							\item Rejects if
							\item Computes $(\hat{u}_{i,j}(\mathbf \alpha))_{i\in[\size]}$
							\item $V$ checks whether $\encode(\mathbf v)_j=\sum_{i\in[\size]}r_i\hat{u}_{i,j}(\mathbf{\alpha})$
						\end{itemize}
				\end{enumerate}
		\item[$\evaluationP:$]				
				\begin{enumerate}
					\item $V$ sends $r_1, \ldots, r_{\size}\in[0, q_0-1]$ and $\mathbf \alpha=(\alpha_0, \ldots, \alpha_{Bdeg})\in[0, q_0-1]^{Bdeg}$
					\item $P$ computes and outputs 
					$$\mathbf{v}=\sum_{i\in[\size]}\phi(q_{1,r})\phi(\matrixV^f_i)\in \FF_q^{\size}$$
					\item $V$ randomly chooses $J\subset[n]$ with $|J|=\Theta(\delta)$ and for each $j\in J$
					\begin{itemize}
						\item If $\mathbf v_j$ is not an integer or $|\mathbf v_j|>...$, $V$ rejects
						\item $V$ checks whether $\encode(\mathbf v_{q, x-\theta})_j=\sum_{i\in[\size]}\phi(q_{1,i})\phi_q(\hat{u}_{i,j}(X))$ and $\phi(y)=\sum_{i\in[\size]}(\mathbf v_{q, x-\theta})_i\phi(q_{2,i})$
					\end{itemize}
				\end{enumerate}
		\end{description}		
	\end{framed}
\end{figure}

	
%%%%%%%%%%%%%%%%%%%%%%%%%%%%%%%%%%%%%%%%%%%%%%%%%%%%%%%%%%%%%%%%%%%%%%%%%%%%%%%%
%%%%%%%%%%%%%%%%%%%%%%%%%%%%%%%%%%%%%%%%%%%%%%%%%%%%%%%%%%%%%%%%%%%%%%%%%%%%%%%%

\subsection{Proximity gaps for infinite rings}
\label{sec:proximity-gap}

\parhead{Informal proximity gap for proposed split encoding}
Consider a projectable linear code $C$ over $\integers$ with a systematic generator matrix $G = [I_{\dim} \ G']$, and a good random choice of $q$. The honest prover will encode a message $m \in \integers^{\dim}$ as $m \| (G'm \bmod q) \in \integers^{\dim} \times \integers_{q}^{\blocklength-\dim}$.

Now consider an arbitrary set of vectors over $v_1,\dotsc,v_{m} \in \integers^{\dim} \times \integers_{q}^{\blocklength-\dim}$, and let $(u_1,\dotsc,u_m) \equiv (v_1,\dotsc,v_{m}) \bmod q$. We know from~\cite[Theorem 4.1]{BKS18} that if there exists $u^\ast \in \set{\sum_{i=1}^{m} \alpha_i u_i \mid \alpha_1,\dotsc,\alpha_m \in \integers_q}$ such that $\reldist{u^\ast}{C} > \delta$, then for any $\epsilon > 0$ such that $\delta-\epsilon < \lambda/3$ it holds that
\[
\Pr_{\alpha \gets \integers_{q}^m} \left[\reldist{\sum_{i=1}^{m} \alpha_i u_i}{C} < \delta - \epsilon\right] < \frac{1}{\epsilon q}
\]
Since for any $\alpha$ we have $\reldist{\sum_{i=1}^{m} \alpha_i v_i}{C} \leq \reldist{\sum_{i=1}^{m} \alpha_i u_i}{C}$, this above result also holds for with respect to the $v_i$.

% We first recall the unique decoding version of the correlated agreement result for linear codes from~\cite{BKS18}.
% \begin{lemma}[Correlated agreement over affine spaces~\cite{BKS18}]
% \label{lemma:correlated-agreement}

% Let $V \subseteq \field^\blocklength$ be a linear space over a finite field $\field$ with $\Delta(C) = \lambda$. Suppose $u^\ast \in \ring^b$ satisfies $\Delta(u^\ast,C) > \delta$, and fix arbitrary $u \in \ring^b$. Let $\challengeset \subseteq \ring$ be an exceptional set. For $\epsilon > 0$ satisfying $\delta - \epsilon < \lambda/3$. Let
% \[
% A = \{\alpha \in \mathcal{C} \mid \distance{u^\ast + \alpha u}{C} < \delta - \varepsilon\}
% \]
% If $|A| > 1/\varepsilon$ then there exist $v, v^\ast \in C$ such that
% \[
% \bigl|\set{i \in [n] \mid (u_i = v_i) \ \wedge \ (u^\ast_i = v^\ast_i)}\bigr| \geq (1 - \delta)\cdot n
% \]
% \parhead{Informal proximity gap for proposed split encoding}
% Consider a projectable linear code $C$ over $\integers$ with a systematic generator matrix $G = [I_{\dim} \ G']$, and a good random choice of $q$. The honest prover will encode a message $m \in \integers^{\dim}$ as $m \| (G'm \bmod q) \in \integers^{\dim} \times \integers_{q}^{\blocklength-\dim}$.

% Now consider an arbitrary set of vectors over $v_1,\dotsc,v_{m} \in \integers^{\dim} \times \integers_{q}^{\blocklength-\dim}$, and let $(u_1,\dotsc,u_m) \equiv (v_1,\dotsc,v_{m}) \bmod q$. We know from~\cite[Theorem 4.1]{BKS18} that if there exists $u^\ast \in \set{\sum_{i=1}^{m} \alpha_i u_i \mid \alpha_1,\dotsc,\alpha_m \in \integers_q}$ such that $\reldist{u^\ast}{C} > \delta$, then for any $\epsilon > 0$ such that $\delta-\epsilon < \lambda/3$ it holds that
% \[
% \Pr_{\alpha \gets \integers_{q}^m} \left[\reldist{\sum_{i=1}^{m} \alpha_i u_i}{C} < \delta - \epsilon\right] < \frac{1}{\epsilon q}
% \]
% Since for any $\alpha$ we have $\reldist{\sum_{i=1}^{m} \alpha_i v_i}{C} \leq \reldist{\sum_{i=1}^{m} \alpha_i u_i}{C}$, this above result also holds for with respect to the $v_i$.

% We first recall the unique decoding version of the correlated agreement result for linear codes from~\cite{BKS18}.
% \begin{lemma}[Correlated agreement over affine spaces~\cite{BKS18}]
% \label{lemma:correlated-agreement}

% Let $V \subseteq \field^\blocklength$ be a linear space over a finite field $\field$ with $\Delta(C) = \lambda$. Suppose $u^\ast \in \ring^b$ satisfies $\Delta(u^\ast,C) > \delta$, and fix arbitrary $u \in \ring^b$. Let $\challengeset \subseteq \ring$ be an exceptional set. For $\epsilon > 0$ satisfying $\delta - \epsilon < \lambda/3$. Let
% \[
% A = \{\alpha \in \mathcal{C} \mid \distance{u^\ast + \alpha u}{C} < \delta - \varepsilon\}
% \]
% If $|A| > 1/\varepsilon$ then there exist $v, v^\ast \in C$ such that
% \[
% \bigl|\set{i \in [n] \mid (u_i = v_i) \ \wedge \ (u^\ast_i = v^\ast_i)}\bigr| \geq (1 - \delta)\cdot n
% \]
% \end{lemma}

%%%%%%%%%%%%%%%%%%%%%%%%%%%%%%%%%%%%%%%%%%%%%%%%%%%%%%%%%%%%%%%%%%%%%%%%%%%%%%%%
%%%%%%%%%%%%%%%%%%%%%%%%%%%%%%%%%%%%%%%%%%%%%%%%%%%%%%%%%%%%%%%%%%%%%%%%%%%%%%%%
\renewcommand*{\bibfont}{\small}
\printbibliography
%%%%%%%%%%%%%%%%%%%%%%%%%%%%%%%%%%%%%%%%%%%%%%%%%%%%%%%%%%%%%%%%%%%%%%%%%%%%%%%%
%%%%%%%%%%%%%%%%%%%%%%%%%%%%%%%%%%%%%%%%%%%%%%%%%%%%%%%%%%%%%%%%%%%%%%%%%%%%%%%%
%%%%%%%%%%%%%%%%%%%%%%%%%%%%%%%%%%%%%%%%%%%%%%%%%%%%%%%%%%%%%%%%%%%%%%%%%%%%%%%%
\end{document}
%%%%%%%%%%%%%%%%%%%%%%%%%%%%%%%%%%%%%%%%%%%%%%%%%%%%%%%%%%%%%%%%%%%%%%%%%%%%%%%%
%%%%%%%%%%%%%%%%%%%%%%%%%%%%%%%%%%%%%%%%%%%%%%%%%%%%%%%%%%%%%%%%%%%%%%%%%%%%%%%%
%%%%%%%%%%%%%%%%%%%%%%%%%%%%%%%%%%%%%%%%%%%%%%%%%%%%%%%%%%%%%%%%%%%%%%%%%%%%%%%%