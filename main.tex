\documentclass[11pt,letterpaper,usenames,dvipsnames]{article}

%%%%%%%%%%%%%%%%%%%%%%%%%%%%%%%%%%%%%%%%%%%%%%%%%%%%%%%%%%%%%%%%%%%%%%%%%%%%%%%%
% PACKAGES
\usepackage{amsmath,amssymb,amsthm}
\usepackage{fullpage}
\usepackage{multirow}
\usepackage{bbm,soul}
\usepackage[usenames,dvipsnames]{xcolor}
\usepackage{lipsum}
\usepackage{makecell}
\usepackage{bm}
\usepackage{braket}
\usepackage{adjustbox,mdframed}
\usepackage[operators,sets,primitives,asymptotics,lambda,keys,ff]{cryptocode}
\usepackage{enumitem}
\usepackage{geometry}
\usepackage{booktabs}
\usepackage[font=footnotesize,labelfont=bf]{caption}
\usepackage{mleftright}
\usepackage{wrapfig}
\usepackage{setspace}
\usepackage{xspace}
\usepackage{algorithm,algorithmic}
\floatname{algorithm}{Protocol}
%\crefname{algorithm}{Protocol}{Protocols}
%\Crefname{algorithm}{Protocol}{Protocols}
\renewcommand{\algorithmiccomment}[1]{// #1}
\renewcommand{\algorithmicrequire}{\textbf{Input:}}
\renewcommand{\algorithmicensure}{\textbf{Output:}}
\usepackage{authblk}

\usepackage{framed}
\usepackage{comment}
\usepackage{placeins} % in the preamble

\usepackage{float}
\usepackage{graphicx}
\graphicspath{ {./figures/} }
\usepackage[T1]{fontenc}
\usepackage{titlesec}
\usepackage{stmaryrd}

\usepackage[backend=biber,giveninits,style=alphabetic,maxalphanames=6,maxnames=6,backref]{biblatex}
\addbibresource{references.bib} 
%\addbibresource{cryptobib/abbrev3.bib} 
%\addbibresource{cryptobib/crypto_crossref.bib}

%%%%%%%%%%%%%%%%%%%%%%%%%%%%%%%%%%%%%%%%%%%%%%%%%%%%%%%%%%%%%%%%%%%%%%%%%%%%%%%%
\newcommand{\plaintitle}{Zinc{+}}
\newcommand{\stylizedtitle}{\textsf{Zinc{+}}}
\PassOptionsToPackage{hyphens}{url}\usepackage{hyperref}
\hypersetup{
    colorlinks=true,
    linkcolor=CadetBlue,
    breaklinks=true,
    citecolor=Periwinkle,      
    urlcolor=Emerald,
    pdftitle={\plaintitle},
    pdfpagemode=FullScreen,
}
\DefineBibliographyStrings{english}{%
  backrefpage = {cited on p.},% originally "cited on page"
  backrefpages = {cited on pp.},% originally "cited on pages"
}
% Remove starred sections from toc
\DeclareRobustCommand{\SkipTocEntry}[5]{} 
\usepackage[capitalize, nameinlink]{cleveref}
\usepackage{tikz}

%%%%%%%%%%%%%%%%%%%%%%%%%%%%%%%%%%%%%%%%%%%%%%%%%%%%%%%%%%%%%%%%%%%%%%%%%%%%%%%%
\newcommand{\psiv}[1]{{\color{Emerald}[#1 -- Psi.]}}
\newcommand{\Albert}[1]{{\color{purple}Albert: #1}}
\newcommand{\Aru}[1]{{\color{blue}Aru: #1}}

\newcommand{\gp}{\mathsf{gp}}
\newcommand{\commit}{\mathsf{Commit}}
\newcommand{\open}{\mathsf{Open}}
\newcommand{\evaluation}{\mathsf{Evaluation}}
\newcommand{\testingP}{\mathsf{Testing Phase}}
\newcommand{\evaluationP}{\mathsf{Evaluation Phase}}
\newcommand{\com}{\mathsf{com}}


\newcommand{\code}{\mathcal{C}}

\newcommand{\polyspace}{(\QQ[X])[\vec Y]}


\newcommand{\encode}{\mathsf{Enc}}


\newcommand{\size}{2^\mu}
\newcommand{\matrixU}{\mathbf{\hat{u}}}
\newcommand{\matrixV}{\mathbf{V}}
\newcommand{\vectorUi}{\mathbf{\hat{u}_i}}
\newcommand{\oracleUi}{\llbracket\mathbf{\hat{u}}_i\rrbracket}

%%%%%%%%%%%%%%%%%%%%%%%%%%%%%%%%%%%%%%%%%%%%%%%%%%%%%%%%%%%%%%%%%%%%%%%%%%%%%%%%
% FORMATTING
%

\interfootnotelinepenalty=10000

\newcommand{\keywords}[1]{\bigskip\par\noindent{\textbf{Keywords\/}: #1}}

%%%%%%%%%%%%%%%%%%%%%%%%%%%%%%%%%%%%%%%%%%%%%%%%%%%%%%%%%%%%%%%%%%%%%%%%%%%%%%%%
% THEOREMS
\theoremstyle{plain} % italics
\newtheorem{itheorem}{Theorem}%[section]
\newtheorem{icorollary}{Corollary}%[section]
\newtheorem{theorem}{Theorem}[section]
\newtheorem{lemma}[theorem]{Lemma}
\newtheorem{proposition}[theorem]{Proposition}
\newtheorem{claim}[theorem]{Claim}
\newtheorem{assumption}{Assumption}
\newtheorem{construction}{Construction}
\theoremstyle{definition} % not italics
\newtheorem{idefinition}{Definition}%[section]
\newtheorem{definition}[theorem]{Definition}
\newtheorem{remark}[theorem]{Remark}
\newtheorem{example}[theorem]{Example}
\theoremstyle{remark} %
\newtheorem{case}{Case}
\crefname{assumption}{Assumption}{Assumptions}
\crefname{step}{Step}{Steps}
\crefname{figure}{Figure}{Figures}

%%%%%%%%%%%%%%%%%%%%%%%%%%%%%%%%%%%%%%%%%%%%%%%%%%%%%%%%%%%%%%%%%%%%%%%%%%%%%%%%
% Make \emph meaningful in italic environments (e.g., when using \theoremstyle{plain})
\renewcommand\eminnershape{\itshape\sffamily}

\newcommand{\emath}[1]{\ensuremath{#1}}
\newcommand{\newc}[2]{\newcommand{#1}{\ensuremath{#2}\xspace}}
\newcommand{\renewc}[2]{\renewcommand{#1}{\ensuremath{#2}\xspace}}

% allow in-line math breaks
\newcommand{\CB}{\allowbreak}

% such that in probability statements
\newcommand{\pST}{\; \middle| \;}
\DeclareMathOperator{\Concat}{\; || \; }
% \DeclareMathOperator{\max}{max}
\newcommand{\Or}{\vee}
\renewcommand{\And}{\wedge}
\newcommand{\Not}{\neg}
\newc{\defeq}{:=}
\newcommand{\Union}{\cup}
\newcommand{\defemph}[1]{\textbf{\emph{#1}}}

% some math macros
\newcommand{\List}[1]{\left[#1\right]}
\newcommand{\SmallList}[1]{[#1]}
\renewcommand{\set}[1]{\ensuremath{\mleft\{#1\mright\}}}
\newcommand{\Range}[2]{\set{#1, \dotsc, #2}}
\newcommand{\OpenOpen}[2]{\left[#1, #2\right]}
\newcommand{\OpenClosed}[2]{\left[#1, #2\right)}
\newcommand{\ClosedOpen}[2]{\left(#1, #2\right]}
\newcommand{\ClosedClosed}[2]{\left(#1, #2\right)}

\newcommand{\Element}[3]{#1_{#2,#3}}

% tuples with in-line math breaks
\newcommand{\tuple}[2]{(#1 ,\CB \dots,\CB #2)}
\newcommand{\pair}[2]{(#1 ,\CB #2)}
\newcommand{\ppair}[2]{\left(#1 ,\CB #2\right)}
\newcommand{\triple}[3]{(#1 ,\CB #2,\CB #3)}
\newcommand{\ptriple}[3]{\left(#1 ,\CB #2,\CB #3\right)}
\newcommand{\quadruple}[4]{(#1 ,\CB #2,\CB #3,\CB #4)}
\newcommand{\pquadruple}[4]{\left(#1 ,\CB #2,\CB #3,\CB #4\right)}
\newcommand{\quintuple}[5]{(#1 ,\CB #2,\CB #3,\CB #4,\CB #5)}
\newcommand{\quintuplenp}[5]{#1 ,\CB #2,\CB #3,\CB #4,\CB #5}
\newcommand{\sextuple}[6]{(#1 ,\CB #2,\CB #3,\CB #4,\CB #5,\CB #6)}
\newcommand{\sextuplenp}[6]{#1 ,\CB #2,\CB #3,\CB #4,\CB #5,\CB #6}
\newcommand{\septuple}[7]{(#1 ,\CB #2,\CB #3,\CB #4,\CB #5,\CB #6,\CB #7)}
\newcommand{\septuplenp}[7]{#1 ,\CB #2,\CB #3,\CB #4,\CB #5,\CB #6,\CB #7}
\newcommand{\octuple}[8]{(#1 ,\CB #2,\CB #3,\CB #4,\CB #5,\CB #6,\CB #7,\CB #8)}
\newcommand{\octuplenp}[8]{#1 ,\CB #2,\CB #3,\CB #4,\CB #5,\CB #6,\CB #7,\CB #8}
\newcommand{\nonuple}[9]{(#1 ,\CB #2,\CB #3,\CB #4,\CB #5,\CB #6,\CB #7,\CB #8,\CB #9)}
\newcommand{\nonuplenp}[9]{#1 ,\CB #2,\CB #3,\CB #4,\CB #5,\CB #6,\CB #7,\CB #8,\CB #9}

% expectation
\newcommand{\prob}[1]{\Pr\left[ #1 \right]}
\newcommand{\Expectation}{\mathbb{E}}

% random variable
\newcommand{\RandomVariable}{X}

% complexity classes
\newcommand{\Class}[1]{\mathsf{#1}}
\newcommand{\NP}{\Class{NP}}
\newcommand{\DTIME}[1]{\Class{DTIME}(#1)}

% types
\newc{\bits}{\{0,1\}}
\newc{\strings}{\bits^{*}}
\newc{\naturals}{\mathbb{N}}
\newc{\cmplx}{\mathbb{C}}
\newc{\reals}{\mathbb{R}}
\newc{\rationals}{\mathbb{Q}}
\newc{\integers}{\mathbb{Z}}
\newc{\integersq}{\mathbb{Z}_{\modulus}}
\newc{\integersp}{\mathbb{Z}_{\pmodulus}}
\newc{\integersalpha}{\mathbb{Z}_{\alpha}}
\newcommand{\integerball}[1]{\integers(#1)}
\newc{\preimagepowers}{P}

\newcommand{\EmptyString}{\varepsilon}

\newcommand{\field}{\mathbb{F}}
\newc{\ff}{\mathbb{F}_{\pmodulus}}
\newcommand{\Pairing}[2]{\PairingSymb(#1,\ #2)}
\newcommand{\fixMSM}{\textrm{f-MSM}}
\newcommand{\varMSM}{\textrm{v-MSM}}
\newcommand{\FFT}{\textrm{FFT}}
\newcommand{\IFFT}{\textrm{IFFT}}

\newcommand{\Max}{\mathsf{Max}}

\newcommand{\innerproduct}[2]{\langle #1, \ #2 \rangle}

\newc{\randpick}{\ {\gets}{\scriptstyle\$}\ }

\newcommand{\Size}[1]{\abs{#1}}
\newcommand{\Time}{t}

\renewc{\secparam}{\lambda}
\newc{\knowledgeerror}{\kappa}
\newcommand{\Deg}{\mathrm{deg}}
\newcommand{\VIEW}{\mathrm{View}}

% attackers
\newcommand{\QuerySampler}{\mathcal{Q}}
\newc{\adversary}{\mathcal{A}}
\newcommand{\adversaryb}{\mathcal{B}}
\newcommand{\adversaryd}{\mathcal{D}}
\newc{\experiment}{\mathsf{Exp}}
\newcommand{\Simulator}{\mathcal{S}}
\newcommand{\PCSimulator}{\mathcal{S}_{\PC}}
\newcommand{\SPCSimulator}{\mathcal{S}_{\mathsf{s}}}
\newcommand{\SqPCSimulator}{\mathcal{S}_{\mathsf{m}}}
\newcommand{\Real}{\mathsf{Real}}
\newcommand{\Ideal}{\mathcal{I}}
\newc{\extractoradversary}{\mathcal{E}_{\adversary}}
\newcommand{\Challenger}{\mathcal{C}}

% oracles
\newcommand{\HashOracle}{\rho}
\newcommand{\Oracle}{\mathcal{O}}
\newcommand{\OracleFamily}{\mathbb{O}}
\newcommand{\SuccessOracles}{\mathsf{S}}

% public parameters
\newcommand{\CRS}{\mathsf{crs}}
\newcommand{\td}{\mathsf{td}}
\newcommand{\PK}{\mathsf{PK}}

% relation, instance, witness
\newcommand{\Relation}{\mathcal{R}}
\newcommand{\Language}{\mathcal{L}}
\newcommand{\Index}{\mathbbm{i}}
\newcommand{\Instance}{\mathbbm{x}}
\newcommand{\InstanceSizeBound}{\mathsf{N}}
\newcommand{\RelationWithSizeBound}{\Relation_{\InstanceSizeBound}}

\newcommand{\ValidInstance}{\Language(\Relation)}
\newcommand{\ValidInstanceWithSizeBound}{\Language(\RelationWithSizeBound)}

\newcommand{\AuxiliaryInp}{\mathbbm{z}}
\newcommand{\AuxiliaryInpDistri}{\mathcal{Z}}
\newcommand{\aux}{\mathsf{aux}}


\newcommand{\polylog}{\mathrm{polylog}}
\newcommand{\hybrid}[1]{\ensuremath{\mathsf{Hyb}_{#1}\xspace}}

\newcommand{\vhl}[1]{{\color{blue} #1}}
\newcommand{\mhl}[1]{\text{\hl{$#1$}}}
\newcommand{\iseq}{\overset{?}{=}}
\newcommand{\isleq}{\overset{?}{\leq}}

% Hash families
\newcommand{\HashFamily}{\mathcal{H}}
\newcommand{\HashIndex}{I}
\newcommand{\HashKey}{I}
\newcommand{\HashIndexSet}{\mathcal{I}}
\newcommand{\Hash}{h}
\newcommand{\HashKeySpace}{\mathcal{I}}

% Transcripts
\newcommand{\TranscriptSet}{\mathcal{X}}
\newcommand{\challengeset}{\mathcal{H}}
\newcommand{\State}{\mathsf{st}}
\newcommand{\Transcript}{\tau}
\newcommand{\Tree}{\textnormal{\textsc{tree}}}
\newcommand{\EmptyTranscript}{\emptyset}
\newcommand{\StatesMultiRounds}[1]{[\State_i]_{i=1}^{\PCRound}}
\newcommand{\QueryTranscript}{\mathsf{qt}}
\newcommand{\Query}{q}
\newcommand{\NumQueries}{Q}
\newcommand{\QuerySet}{\mathcal{Q}}
\newcommand{\ChallengeSetSize}{p}
\newcommand{\Image}{\mathsf{Image}}

% Algebras
\newc{\ring}{\mathcal{R}}
\newc{\ideal}{\mathfrak{n}}
\newc{\ringdegree}{n}
\newc{\modulus}{q}
\newc{\pmodulus}{M}
\newc{\modq}{\bmod\modulus}
\newc{\modp}{\bmod\pmodulus}
\newc{\fring}{\ring_{\modulus}}
\newc{\pring}{\ring_{\pmodulus}}
\newc{\fringinvertiblesubset}{\ring_{\modulus}^{\times}}
\newc{\fringmultgroup}{\fring^{\times}}
\newcommand{\automorphism}[1]{\emath{\sigma_{-1}(#1)}}
\newcommand{\ltwonorm}[1]{\norm{#1}_{2}}
\newcommand{\linfinity}{\ell_\infty}
\newcommand{\linfinitynorm}[1]{\norm{#1}_{\infty}}

% Codes
\renewc{\dist}{\mathsf{dist}}
\renewc{\dim}{\mathsf{dim}}
\newc{\blocklength}{\mathsf{n}}
\newcommand{\reldist}[2]{\Delta\left(#1,#2\right)}

% Footnotes
\newcommand{\astfootnote}[1]{%
\let\oldthefootnote=\thefootnote%
\setcounter{footnote}{0}%
\renewcommand{\thefootnote}{\fnsymbol{footnote}}%
\footnote{#1}%
\addtocounter{footnote}{-1}%
\let\thefootnote=\oldthefootnote%
}

\newcommand\blfootnote[1]{%
  \begingroup
  \renewcommand\thefootnote{}\footnote{#1}%
  \addtocounter{footnote}{-1}%
  \endgroup
}

% Better paragraphs headers
\newcommand{\parhead}[1]{\medskip\noindent\textbf{#1}.\quad}

\newcommand{\newdefs}[1] {\setlength{\fboxsep}{1pt}\colorbox{gray!20}{\(#1\)}}


%general formatting
\newcommand{\pcvarstyle}[1]{\mathsf{#1}}
\newcommand{\continue}{{\Huge{\hl{$\cdots$}}}}
% \renewcommand{\case}[1]{\medskip\noindent{\fbox{Case #1:}}}
% \newcommand{\ncase}[1]{\medskip\noindent{\fbox{#1:}}}
% \newcommand{\ngame}[1]{\medskip\noindent{\fbox{Game $\game{#1}$:}}}
\newcommand{\mhyph}{\text{-}}
\newcommand{\ncase}[1]{\medskip\noindent{\textbf{#1:}}}
\newcommand{\ngame}[1]{\medskip\noindent{\textbf{Game $\game{#1}$:}}}
\newcommand{\conclude}{\medskip\noindent{}}
\newcommand{\myskip}{-0.16\baselineskip}

% Make vectors bold
\let\vec\mathbf

% General mathematics
\newcommand{\range}[2] {[#1 \, .. \, #2]}
\newcommand{\SD}{\Delta}
\newcommand{\smallset}[1] {\{#1\}}
\newcommand{\bigset}[1] {\left\{#1\right\}}
\newcommand{\GRP} {\mathbb{G}}
\newcommand{\HRP} {\mathbb{H}}
%\newcommand{\pair} {\hat{e}}
\newcommand{\brak}[1] {\left(#1\right)}
\newcommand{\sbrak}[1] {(#1)}
\newcommand{\alg}[1] {\pcalgostyle{#1}}
\newcommand{\image} {\operatorname{im}}
\newcommand{\myland} {\,\land\,}
\newcommand{\mylor} {\,\lor\,}
\newcommand{\w}{\omega}
\newcommand{\const}{\pcpolynomialstyle{const}}
\newcommand{\p}[1]{\pcpolynomialstyle{#1}}
\newcommand{\ptx}{\p{t_{X^0}}}
\newcommand{\ptxsim}{\p{\tilde{t}_{X^0}}}
\newcommand{\ev}[1]{\widetilde{\pcpolynomialstyle{#1}}}
\newcommand{\maxconst}{\pcvarstyle{max}}
\newcommand{\numberofconstrains}{\pcvarstyle{n}}
\newcommand{\noofc}{\numberofconstrains}
\newcommand{\noofw}{\pcvarstyle{m}}
\newcommand{\dconst}{\pcvarstyle{d}}
\newcommand{\multconstr}{\pcvarstyle{n}}
\newcommand{\linconstr}{\pcvarstyle{Q}}
\newcommand{\expected}[1]{\mathbb{E}\left[#1\right]}
\newcommand{\infrac}[2]{#1 / #2}
\newcommand{\eps}{\varepsilon}
\DeclareMathOperator{\SPAN}{Span}
\newcommand{\maxdeg}{\pcvarstyle{max}}
%polynomials
\newcommand{\pf}{\p{pf}}
\newcommand{\pF}{\p{F}}
\newcommand{\pa}{\p{a}}
\newcommand{\pb}{\p{b}}
\newcommand{\pc}{\p{c}}
\newcommand{\pz}{\p{z}}
\newcommand{\pt}{\p{t}}
\newcommand{\pR}{\p{R}}
\newcommand{\pr}{\p{r}}
\newcommand{\ptlo}{\p{t_{lo}}}
\newcommand{\ptmid}{\p{t_{mid}}}
\newcommand{\pthi}{\p{t_{hi}}}
\newcommand{\pS}{\p{S}}
\newcommand{\pT}{\p{T}}
\newcommand{\pta}{\tilde{\p{a}}}
\newcommand{\ptb}{\tilde{\p{b}}}
\newcommand{\ptc}{\tilde{\p{c}}}
\newcommand{\ptz}{\tilde{\p{z}}}
\newcommand{\ptZH}{\tilde{\p{Z_H}}}
\newcommand{\vX}{\vec{X}}
\newcommand{\vB}{\vec{B}}
\newcommand{\va}{\vec{a}}
\newcommand{\vb}{\vec{b}}
\newcommand{\vc}{\vec{c}}
\newcommand{\tzkproof}{\pi}

%reduction
\newcommand{\tb}{\tilde{b}} 
\newcommand{\tw}{\tilde{w}} 

% bilinear maps

\newcommand{\bmap}[2] {\left[#1\right]_{#2}}
\newcommand{\gone}[1] {\bmap{#1}{1}}
\newcommand{\gtwo}[1] {\bmap{#1}{2}}
\newcommand{\gi} {\iota}
\newcommand{\gtar}[1] {\bmap{#1}{T}}
\newcommand{\grpgi}[1] {\bmap{#1}{\gi}}
\newcommand{\gnone}[1]{\left[#1\right]}

\newcommand{\msg}[1]{\mathtt{#1}}

% zero knowledge
\newcommand{\oracleo}{\mathsf{O}}
\newcommand{\oraclesrs}{\oracleo_{srs}}
\newcommand{\oraclec}{\oracleo_\cdv}
% for srs updatability
\newcommand{\intent}{\msg{intent}}
\newcommand{\update}{\msg{update}}
\newcommand{\setup}{\msg{setup}}
\newcommand{\final}{\msg{final}}
\newcommand{\verifysrs}{\verify_{srs}}
\newcommand{\srsupdate}{\pcalgostyle{Update}}
\newcommand{\crs}{\pcvarstyle{crs}}
\newcommand{\srs}{\pcvarstyle{srs}}
%\newcommand{\td}{\pcvarstyle{td}}
\newcommand{\ip}[2]{\left\langle #1, #2\right\rangle}
\newcommand{\zkproof}{\pi}
\newcommand{\proofsystem}{\pcschemestyle{\Psi}}
\newcommand{\ps}{\proofsystem}
\newcommand{\psfs}{\proofsystem_\fs}
% \newcommand{\ps}{\proofsystem}
\newcommand{\nuppt}{\pcmachinemodelstyle{NUPPT}}
\newcommand{\ro}{\mathcal{H}}
\newcommand{\rof}[2]{\mathbf{\Omega}_{#1, #2}}
\newcommand{\trans}{\pcvarstyle{trans}}
\newcommand{\tr}{\pcvarstyle{tr}}
\newcommand{\instsize}{\pcvarstyle{n}}
\newcommand{\KG} {\mathsf{K}}
\newcommand{\kcrs} {\KG_{\crs}}
\newcommand{\fs}{\pcalgostyle{FS}}
\newcommand{\sigmaprot}{\pcalgostyle{\Sigma}}
\newcommand{\se}{\pcvarstyle{se}}
\newcommand{\snd}{\pcvarstyle{snd}}
\newcommand{\zk}{\pcvarstyle{zk}}


%rewinding---tree of transcripts
\newcommand{\pcboolstyle}[1]{\mathtt{#1}}
\newcommand{\treebuild}{\pcalgostyle{TreeBuild}}
\newcommand{\tree}{\pcvarstyle{T}}
\newcommand{\counter}{\pcvarstyle{counter}}


%PLONK related
\newcommand{\pcschemestyle}[1]{\mathsf{#1}}
\newcommand{\plonkprot}{\pcschemestyle{P}}
\newcommand{\plonkprotfs}{\pcschemestyle{P}_\fs}
\newcommand{\sonicprot}{\pcschemestyle{S}}
\newcommand{\sonicprotfs}{\pcschemestyle{S}_\fs}
\newcommand{\marlinprot}{\pcschemestyle{M}}
\newcommand{\marlinprotfs}{\pcschemestyle{M}_\fs}
\newcommand{\selector}[1]{\pcvarstyle{q_{#1}}}
\newcommand{\selmulti}{\selector{M}}
\newcommand{\selleft}{\selector{L}}
\newcommand{\selright}{\selector{R}}
\newcommand{\seloutput}{\selector{O}}
\newcommand{\selconst}{\selector{C}}
\newcommand{\chz}{\mathfrak{z}}
\newcommand{\ochz}{{\omega \mathfrak{z}}}
\newcommand{\reduction}{\rdv}
\newcommand{\tdv}{\pcadvstyle{T}}
\newcommand{\ch}{\pcvarstyle{ch}}

\newcommand{\game}[1]{\pcalgostyle{G}_{#1}}

\newcommand{\lag}{\p{L}}
\newcommand{\pubinppoly}{\p{PI}}

\newcommand{\ZERO}{\p{Z}}
%\newcommand{\pcsetstyle}[1]{\mathrm{#1}}
\newcommand{\HHH}{\mathsf{H}}
\newcommand{\KKK}{\mathsf{K}}

% general complexity theory
% \newcommand{\RND}[1]{\pcalgostyle{RND}(#1)}
\newcommand{\RND}[1]{\pcvarstyle{R}(#1)}
\newcommand{\RELGEN}{\mathcal{R}}
\newcommand{\REL}{\mathsf{REL}}
\newcommand{\LANG}{\mathcal{L}}
\newcommand{\inp}{\mathbbm{x}}
\newcommand{\wit}{\mathbbm{w}}
\newcommand{\class}[1]{\mathfrak{#1}}
\newcommand{\ig}{\pcalgostyle{IG}}
\newcommand{\accProb}{\event{acc}}
\newcommand{\waccProb}{\event{\widetilde{acc}}}
\newcommand{\frkProb}{\event{frk}}
\newcommand{\extProb}{\event{ext}}
\newcommand{\ssndProb}{\event{ssnd}}
\newcommand{\FS}{\pcalgostyle{FS}} % Fiat-Shamir transform
\renewcommand{\aux}{\pcvarstyle{aux}} %auxiliary input
% \newcommand{\dlog}{\pcvarstyle{dlog}}
\newcommand{\vereq}{\p{ve}}

%Commitment schemes
\newcommand{\COM}{\pcschemestyle{C}}
\newcommand{\PCOM}{\pcschemestyle{PC}}
\newcommand{\PCOMp}{\pcschemestyle{PC}_{\plonkprot}}
\newcommand{\PCOMs}{\pcschemestyle{PC}_{\sonicprot}}
\renewcommand{\com}{\pcalgostyle{Com}}
\newcommand{\op}{\pcalgostyle{Op}}
\renewcommand{\open}{\op}

\newcommand{\committer}{\pcalgostyle{C}}
\newcommand{\receiver}{\pcalgostyle{R}}

%Plonk and Sonic
\newcommand{\ur}[1]{{#1\text{-}\mathsf{ur}}}

\newcommand{\plonk}{\ensuremath{\textnormal{\textsf{Plonk}}}}
\newcommand{\marlin}{{\ensuremath{\textnormal{\textsf{Marlin}}}}}
\newcommand{\sonic}{{\ensuremath{\textnormal{\textsf{Sonic}}}}}
\newcommand{\groth}{\ensuremath{\textsc{Groth16}}}
\newcommand{\plonkmod}{\ensuremath{\plonk^\star}}
\newcommand{\plonkint}{\ensuremath{\plonk^\star}}
\newcommand{\polyprot}{\pcalgostyle{poly}}
\newcommand{\plonkintpoly}{\plonkint_\polyprot}
% \newcommand{\sonic}{\textsc{Sonic}}
\newcommand{\maxdegree}{\pcvarstyle{N}}

\newcommand{\dlog}{\pcvarstyle{dlog}}
\newcommand{\ldlog}{\pcvarstyle{ldlog}}


%reductions
\newcommand{\extss}{\ext_\sfss}
\newcommand{\extse}{\ext_\se}
\newcommand{\extt}{\ext_{\pcvarstyle{tree}}}
\newcommand{\compass}{\mathtt{C}}
\newcommand{\ks}{\pcvarstyle{ks}}
\newcommand{\sfss}{\pcvarstyle{ss}}
\newcommand{\rdvks}{{\rdv_\ks}}
\newcommand{\rdvs}{{\rdv_\pcvarstyle{s}}}
\newcommand{\rdvdlog}{{\rdv_\dlog}}
\newcommand{\rdvldlog}{{\rdv_\ldlog}}
\newcommand{\rdvse}{{\rdv_\se}}
\newcommand{\rdvss}{{\rdv_\sfss}}
\newcommand{\rdvur}{\rdv_\pcvarstyle{ur}}

\newcommand{\env}{\pcadvstyle{E}}
\newcommand{\zdv}{\pcadvstyle{Z}}

\newcommand{\advse}{\adv}
\newcommand{\advss}{{\adv_\sfss}}
\newcommand{\bdvfs}{\bdv_{\FS}}

\newcommand{\epsss}{\eps_\pcvarstyle{f}}
\newcommand{\epsur}{\eps_\pcvarstyle{ur}}
\newcommand{\epsh}{\eps_{\pcvarstyle{hid}}}
\newcommand{\epsk}{\eps_{\pcvarstyle{k}}}
\newcommand{\epsbind}{\eps_\pcvarstyle{bind}}
\newcommand{\epsbinding}{\eps_\pcvarstyle{bind}}
\newcommand{\epsid}{\eps_{\pcvarstyle{id}}}
\newcommand{\epsop}{\eps_\pcvarstyle{op}}
\newcommand{\epss}{\eps_\pcvarstyle{s}}
\newcommand{\epsfor}{\eps_\pcvarstyle{f}}
\newcommand{\epsbatch}{\eps_\pcvarstyle{btch}}
\newcommand{\epsdlog}{\eps_\dlog}
\newcommand{\epsldlog}{\eps_\ldlog}
\newcommand{\epsuber}{\eps_{\pcvarstyle{uber}}}
\newcommand{\epszk}{\eps_{\pcvarstyle{zk}}}
\newcommand{\epsro}{\eps_{\ro}}

%selection polynomials
\newcommand{\vql}{\vec{q_{L}}}
\newcommand{\vqr}{\vec{q_{R}}}
\newcommand{\vqm}{\vec{q_{M}}}
\newcommand{\vqo}{\vec{q_{O}}}
\newcommand{\vx}{\vec{x}}
\newcommand{\vqc}{\vec{q_{C}}}

%errors
\newcommand{\err}{Err}
\newcommand{\errur}{\err_{ur}}
\newcommand{\errss}{\err_\sfss}
\newcommand{\errfrk}{\err_\frkProb}


%forking
\newcommand{\forking}{\pcalgostyle{F}}
\newcommand{\genforking}{\pcalgostyle{GF}}

%moving proofs and instances
\newcommand{\MoveInstanceForward}{\pcalgostyle{MoveInstanceForward}}
\newcommand{\MoveInstanceBackward}{\pcalgostyle{MoveInstanceBackward}}
\newcommand{\MoveProofForward}{\pcalgostyle{MoveProofForward}}
\newcommand{\MoveProofBackward}{\pcalgostyle{MoveProofBackward}}

%% Fields and other math
\newcommand{\HH}{\mathbb{H}}
\newcommand{\BB}{\{0,1\}}
\newcommand{\G}{\mathbb{G}}
\newcommand{\GT}{\mathbb{G}_T}
%Luca's commands and macros
\newcommand{\K}{\mathbb{K}}
\newcommand{\Q}{\mathbb{Q}}
\newcommand{\B}{\mathbb{B}}
\newcommand{\F}{\mathbb{F}}

%% hash, epsilons, protcools
\newcommand{\HASH}{\pcalgostyle{Hash}}
\newcommand{\vareps}{\varepsilon}
\newcommand{\varepscommit}{\vareps_{\text{commit}}}
\newcommand{\varepsquery}{\vareps_{\text{query}}}
\newcommand{\prot}{\pcalgostyle{\Pi}}
\newcommand{\protfs}{\pcalgostyle{\Pi}_{\FS}}
\newcommand{\proverfs}{\prover_{\FS}}
\newcommand{\verifierfs}{\verifier_{\FS}}

%colors
\definecolor{darkmagenta}{rgb}{0.5,0,0.5}
\definecolor{lightmagenta}{rgb}{1,0.85,1}
\definecolor{lightmagenta}{rgb}{0.9,0.9,0.9}
\definecolor{darkred}{rgb}{0.7,0,0}
\definecolor{blueish}{rgb}{0.1,0.1,0.5}
\definecolor{pinkish}{rgb}{0.9,0.8,0.8}
\definecolor{darkgreen}{rgb}{0,0.6,0}
\definecolor{lightgreen}{rgb}{0.85,1,0.85}
\definecolor{skyblue}{rgb}{0.3,0.9,0.99}
\definecolor{nicepurple}{rgb}{0.40784, 0.05490, 0.29412}

%\newcommand{\pcschemestyle}[1]{\mathbf{#1}}

\usepackage[most]{tcolorbox}

%comments
\DeclareRobustCommand{\markulf}[2]  {{\color{darkmagenta}\hl{\scriptsize\textsf{Markulf #1:} #2}}}
\DeclareRobustCommand{\michals}[2]  {{\color{blueish}{\scriptsize\textsf{Michal #1:}} #2}}
\newcommand{\chaya}[1]{\textcolor{magenta}{[{\footnotesize {\bf Chaya:} { {#1}}}]}}
\newcommand{\hamid}[1]{\textcolor{blue}{[{\footnotesize {\bf Hamid:} { {#1}}}]}}
\newcommand{\task}[2]{\todo[author=\textbf{Task},inline]{({\textit{#1}}) #2}}
% \newcommand{\task}[2] {\xcommenti{Task}{#1}{#2}}
% \DeclareRobustCommand{\task}[2]  {{\color{black}\sethlcolor{yellow}\hl{\textsf{TASK #1:} #2}}}
\DeclareRobustCommand{\changedm}[1] {{\color{violet} #1}}



% English Abbreviations
\usepackage{xspace}
\newcommand{\ie}{\text{i.e.}\xspace}
\newcommand{\etal}{\text{et al.}\xspace}
\newcommand{\naive}{\text{na\"ive}\xspace}
\newcommand{\wrt}{\text{w.r.t.}\xspace}
\newcommand{\whp}{\text{w.h.p.}\xspace}
\newcommand{\eg}{\text{e.g.}\xspace}
\newcommand{\cf}{\text{cf.}\xspace}
\newcommand{\visavis}{\text{vis-\`a-vis}\xspace}
\newcommand{\aka}{\text{a.k.a.}\xspace}
\newcommand{\ala}{\text{\`{a} la}\xspace}
\newcommand{\iid}[0]{\text{i.i.d.}\xspace}




\newcommand{\plonkytwoconstraint}{(\mathcal{P}_\inp, \mathcal{Q}_\inp, \sigma_\inp, \p{PI}=\emptyset, \p{r}_\inp, \p{r}', \p{\ell}_\inp, \p{n}_\inp,\p{t}, \FF, \mathbb{K})}

\newcommand{\plonkytwoconstraintnox}{(\mathcal{P}, \mathcal{Q}, \sigma, \p{PI}=\emptyset, \p{r}, \p{r}', \p{\ell}, \p{n},\p{t}, \FF, \mathbb{K})}


\newcommand{\partf}[2]{\bar{f}_{#1}(\omega^{#2})}
\newcommand{\partfx}[1]{\bar{f}_{#1}(X)}
\newcommand{\partfxk}[1]{\bar{f}_{k{#1}}(X)}
\newcommand{\partfxkfull}[2]{\bar{f}_{{#2}{#1}}(X)}
\newcommand{\partfxkfullx}[2]{\bar{f}_{{#2}{#1}}(x)}
\newcommand{\partffull}[3]{\bar{f}_{#1}(\tau_{1{#3}}, \omega^{#2})}
\newcommand{\partfxfull}[2]{\bar{f}_{#1}(\tau_{1{#2}},X)}


\newcommand{\partg}[2]{\bar{g}_{#1}(\omega^{#2})}
\newcommand{\partgx}[1]{\bar{g}_{#1}(X)}
\newcommand{\partgxk}[1]{\bar{g}_{k{#1}}(X)}
\newcommand{\partgxkfull}[2]{\bar{g}_{{#2}{#1}}(X)}
\newcommand{\partgxkfullx}[2]{\bar{g}_{{#2}{#1}}(x)}
\newcommand{\partgfull}[3]{\bar{g}_{#1}(\tau_{1{#3}}, \omega^{#2})}
\newcommand{\partgxfull}[2]{\bar{g}_{#1}(\tau_{1{#2}},X)}

\newcommand{\roundonemessage}{(\p{a}_i(X), \p{b}_i(X))_{i\in [\p{r}]}}
\newcommand{\roundonecha}{(\beta_i, \gamma_i)_{i\in [\p{t}]}}
\newcommand{\roundtwomessage}{(\p{z}_k(X), \pi_{ki}(X))_{k\in [\p{t}], i\in [\p{s}-1]}}
\newcommand{\roundtwocha}{(\alpha_i)_{i\in [\p{t}]}}
\newcommand{\roundonepartialtr}{ ((\p{a}_i(X),)_{i\in [\p{r}]}, (\beta_i, \gamma_i)_{i\in [\p{t}]})}
\newcommand{\roundonepartialtronebetagamma}[1]{ ((\p{a}_i(X))_{i\in [\p{r}]}, (\beta_{#1}, \gamma_{#1}))}
\newcommand{\roundtwopartialtr}{ ((\p{a}_i(X))_{i\in [\p{r}]}, \roundonecha, \roundtwomessage, \roundtwocha)_{i\in [\p{t}]})}
\newcommand{\roundtwopartialtronealpha}[1]{ ((\p{a}_i(X))_{i\in [\p{r}]}, (\beta_i, \gamma_i)_{i\in [\p{t}]}, (\p{z}_k(X), \pi_{ki}(X))_{k\in [\p{t}], i\in [\p{s}-1]}, \alpha_{#1})}
\newcommand{\polyroundtwofull}[1]{\p{d}(\inp, \tau_{2{#1}}, X)}
\newcommand{\polyroundtwofulleval}[2]{\p{d}(\inp, \tau_{2{#1}}, {#2})}


\newcommand{\RS}{\pred{RS}}
\newcommand{\bbF}{\FF}
\renewcommand{\defeq}[0]{\ensuremath{{\;\vcentcolon=\;}}\xspace}

\newcommand{\idx}{\mathbbm{i}}
\newcommand{\permrelation}{\REL_{\mathbf{RPerm}}}
\newcommand{\permindex}{(\p{r},  \sigma)}

\newcommand{\turboplonkrelation}{{\REL_{\mathbf{RTurboPlonk}}}}
\newcommand{\plonkyrelation}{{\REL_{\mathbf{ROPlonky}}}}
\newcommand{\plonkytworelation}{{\REL_{\mathbf{RPlonky2}}}}

\newcommand{\oracle}[1]{\llbracket
{#1}\rrbracket
}

\newcommand{\LT}{\p{LT}}
\newcommand{\turboplonkidx}{( \mathcal{P}, \mathcal{Q}, \sigma, \p{PI}, \p{r}, \p{\ell})}
\newcommand{\turboplonkconstraint}{\turboplonkidx}
\newcommand{\plonkyindex}{( \mathcal{P}, \mathcal{Q}, H, \sigma, \p{PI}, \p{r},\p{r}', \ell, \p{t})}
\newcommand{\plonkyindexexpanded}{(\FF, \mathbb{K}, \omega, \mathcal{P}, \mathcal{Q}, \sigma, \p{PI}, \p{r},\p{r}', \p{\ell},\p{t},\hash)}

\newcommand{\parttr}[1]{\mathcal{T}_{#1}}
\newcommand{\idxmaxf}{\p{idx_{max}}(F)}
\newcommand{\idxmaxg}{\p{idx_{max}}(G)}
\newcommand{\idxminf}{\p{idx_{min}}(F)}
\newcommand{\idxming}{\p{idx_{min}}(G)}
\newcommand{\roundonechanoidx}{(\beta, \gamma)}
\newcommand{\llex}{\leq_{\p{llex}}}
\newcommand{\pdiag}{\p{pdiag}}


\newcommand{\Ind}{\p{Ind}}


\newcommand{\plonky}{\p{Plonky}}
\newcommand{\plonkyindexer}{\Ind_{\plonky}}
\newcommand{\plonkyprover}{\prover_{\plonky}}
\newcommand{\plonkyverifier}{\verifier_{\plonky}}

\newcommand{\plonkyldtoracle}{\p{OPlonky}\xspace}
\newcommand{\plonkyldtindexer}{\Ind_{\plonkyldtoracle}}
\newcommand{\plonkyldtprover}{\prover_{\plonkyldtoracle}}
\newcommand{\plonkyldtverifier}{\verifier_{\plonkyldtoracle}}

\newcommand{\plonkyfs}[1]{\prot_{\p{Plonky2}, \FS}({#1})}
\newcommand{\plonkyproverfs}[1]{\prover_{\p{Plonky2}, \FS}({#1})}
\newcommand{\plonkyverifierfs}[1]{\verifier_{\p{Plonky2}, \FS}({#1})}


\newcommand{\plonkysmall}{\prot_{\p{rounds123}}}
\newcommand{\plonkysmallprover}{\prover_{\p{rounds123}}}
\newcommand{\plonkysmallverifier}{\verifier_{\p{rounds123}}}



\newcommand{\quotient}[3]{\p{quotient}({#1},{#2},{#3})}

\newcommand{\KK}{\mathbb{K}}

% \newcommand{\ro}{\mathsf{H}}
\newcommand{\zinc}{\mathsf{Zinc}}

\newcommand{\batchedfri}{\p{batchFRI}}
\newcommand{\ldtrelation}{\mathbf{PerfLDT}}
\newcommand{\ppp}{\p{gp}}
\newcommand{\pppc}{\mathsf{pp}_{\mathsf{Com}}}
\newcommand{\idxacc}{\idx_{\mathsf{acc}}}
\newcommand{\inpacc}{\inp_{\mathsf{acc}}}
\newcommand{\witacc}{\wit_{\mathsf{acc}}}

\newcommand{\permarg}{\prot_{\permrelation}}
\newcommand{\permargprover}{\prover_{\permrelation}}
\newcommand{\permargverifier}{\verifier_{\permrelation}}
\newcommand{\permargindexer}{\Ind_{\permrelation}}

\newcommand{\doomedldt}{\cD_{0}}
\newcommand{\doomedldtprox}{\cD_{\delta}}
\newcommand{\doomedldtks}{\cD_{0,\p{ks}}}
\newcommand{\doomedldtproxks}{\cD_{\delta,  \p{ks}}}

\newcommand{\union}{\ensuremath{\cup}\xspace}
\newcommand{\intersect}{\ensuremath{\cap}\xspace}
\newcommand{\pow}[1]{\ensuremath{^{(#1)}}\xspace}
\newcommand{\fin}{\pred{f}}
\newcommand{\tQ}{\ensuremath{\widetilde{Q}}\xspace}



\newcommand{\polytranscripts}[1]{\text{PolyTr}(#1)}
\newcommand{\mapstranscript}[1]{\text{Words}(#1)}
\newcommand{\extmapstranscript}[1]{\text{ExtWords}(#1)}


\newcommand{\epsfrirbr}{\ensuremath{\eps_{\p{rbr}}^{\p{FRI}}}\xspace}
\newcommand{\epsbatchfrirbr}{\ensuremath{\eps_{\p{rbr}}^{\p{bFRI}}}\xspace}
\newcommand{\epsfrirbrk}{\ensuremath{\eps_{\p{rbr-k}}^{\p{FRI}}}\xspace}
\newcommand{\epsbatchfrirbrk}{\ensuremath{\eps_{\p{rbr-k}}^{\p{bFRI}}}\xspace}
\newcommand{\epsfrifs}{\ensuremath{\eps_{\p{fs}}^{\p{FRI}}}\xspace}
\newcommand{\epsbatchfrifs}{\ensuremath{\eps_{\p{fs}}^{\p{bFRI}}}\xspace}
\newcommand{\epsfrifsq}{\ensuremath{\eps_{\p{fs-q}}^{\p{FRI}}}\xspace}
\newcommand{\epsbatchfrifsq}{\ensuremath{\eps_{\p{fs-q}}^{\p{bFRI}}}\xspace}

\newcommand{\?}{\ensuremath{\stackrel{?}{=}}\xspace}
\newcommand{\oplonky}{\ensuremath{\plonkyldtoracle}\xspace}


\newcommand{\oracleagreementdelta}{{\p{OCoAgg}(\delta)}}
\newcommand{\oracleagreementzero}{{\p{OCoAgg}(0)}}
\newcommand{\ldtrelationprox}[1]{{\mathbf{CoAgg({#1})}}}
\newcommand{\ldtrelationproxplain}{{\mathbf{CoAgg}}}
\newcommand{\protcoagg}[1]{\prot^{\p{OCoAgg}({#1})}}
\newcommand{\protcoaggactualrel}[1]{{\p{CoAgg}({#1})}}

\newcommand{\idxcorrelatedagreement}{(\FF, D, d, \delta, r)}


  \newcommand{\protcompiled}{ \prot_{\p{compiled}}  } \newcommand{\indexcompiled}{\Ind_{\p{compiled}}}
  \newcommand{\provercompiled}{\prover_{\p{compiled}}}  \newcommand{\verifiercompiled}{\verifier_{\p{compiled}}}

  \newcommand{\plonkyhIOP}{\p{Plonky2\-hIOP}}


    \newcommand{\ca}{\p{CA}}
    \newcommand{\idxinp}{\idx,\inp}


\newcommand{\rank}{\rho}
\newcommand{\Free}{\mathbb{FR}_\rank}
\newcommand{\Freenum}[1]{F_{#1}}
\newcommand{\Id}{\p{Id} }
\newcommand{\mle}[1]{{\p{mle}[{#1}]}}
\newcommand{\hypercube}[1]{\{0,1\}^{#1}}
\newcommand{\FCCSunrolled}{(\rho, I=(f_1, \ldots, f_k), m,n,\ell,t,q,d), (M_0,\ldots, M_{t-1}, c_0,\ldots, c_{q-1}))}

%%ethSTARK
\newcommand{\traceevaldomain}{\pcvarstyle{D}}
\newcommand{\generator}{\pcvarstyle{g}}


\newcommand{\vect}[1]{\mathbf{#1}}
\newcommand{\bigtable}{\vect{t}}
\newcommand{\sizetable}{N}
\newcommand{\smalltable}{\vect{a}}
\newcommand{\smalltabletwo}{\vect{b}}
\newcommand{\smalltablethree}{\vect{c}}
\newcommand{\sizesmalltable}{m}
%\newcommand{\setvec}[1]{\left\{ #1 \right\}}

%%lookup tuples
\newcommand{\sizes}{(\sizetable, \sizesmalltable)}

\newcommand{\lookuptuple}{(\sizes, \bigtable; \smalltable)}

\newcommand{\indxedlookuptuple}{(\sizes, \bigtable; \smalltable, \tau)}

\newcommand{\indxedlookuptupleone}{(\sizes, \bigtable; \smalltable_1, \tau_1)}

\newcommand{\indxedlookuptupletwo}{(\sizes, \bigtable; \smalltable_2, \tau_2)}

\newcommand{\sizesnew}{(\sizetable+\sizesmalltable, \sizesmalltable)}

\newcommand{\indxedlookuptupleprime}{(\sizesnew, \bigtable'; \smalltable, \tau')}

\newcommand{\commatlooktuple}[1]{((N,m), \overline{M_{#1}}, \overline{\smalltable_{#1}}, \overline{\bigtable} ;  M_{#1}, \smalltable_{#1}, \bigtable)}

%%relations

\newcommand{\protsetequality}{\prot_{\p{set-eq}}}
\newcommand{\relsetequality}{\REL_{\p{set-equal}}}


    %\newcommand{\todo}[1]{}
    %\newcommand{\albert}[1]{}
    %\newcommand{\hendrik}[1]{}
    %\newcommand{\katy}[1]{}
    %\newcommand{\luca}[1]{}




%\newcommand{\ilia}[1]{\textcolor{nicepurple}{#1}}
%\newcommand{\nacho}[1]{\textcolor{orange}{#1}}
%\newcommand{\matteo}[1]{\textcolor{red}{#1}}
%\newcommand{\nick}[1]{\textcolor{violet}{#1}}



\newcommand{\protperm}{\prot_{\p{perm}}}
\newcommand{\relperm}{\REL_{\p{perm}}}
\newcommand{\tildepol}[1]{\widetilde{#1}}
\newcommand{\ind}{\mathbb{I}}

\newcommand{\Rccs}{\REL_{\p{cR1CS}}}
\newcommand{\Rcs}{\REL_{\p{R1CS}}}
\newcommand{\Rcrcs}{\REL_{\p{acc}}}
\newcommand{\Rcsa}{\Rcrcs(\textbf{r})}
\newcommand{\Rms}{\REL_{\p{ms}}}
\newcommand{\Rsps}{\REL_{\p{sps}}}

%%variables
\newcommand{\bb}{\vec{b}}
\newcommand{\yy}{\vec{y}}
\newcommand{\xx}{\vec{x}}
\newcommand{\XX}{\vec{X}}
\newcommand{\EE}{\mathbf{E}}
\newcommand{\zz}{\vec{z}}
\renewcommand{\ZZ}{\mathbb{Z}}

\renewcommand{\ss}{\vec{s}}
\renewcommand{\rr}{\vec{r}}
\newcommand{\YY}{\mathbf{Y}}
\newcommand{\WW}{\mathbf{W}}

%\newcommand{\lEE}{\lin(\mathbf{E})}

\newcommand{\TT}{\mathbf{T}}
\newcommand{\multilinmle}[1]{\widetilde{#1}}
\newcommand{\multilin}{\mathsf{ml}}

\newcommand{\bWW}{{\widebar{\WW}}}
\newcommand{\bEE}{{\widebar{\EE}}}
\newcommand{\bTT}{\widebar{\TT}}



\newcommand{\MM}{\mathbf{M}}
\newcommand{\bMM}{\widebar{\mathbf{M}}}
%%lookup relations
%%normal lookup
\newcommand{\rellookup}{\mathsf{Look}_{\mathsf{gp}}}
\newcommand{\relindexedlookup}{\REL_{\p{IdxLook}}}
%%algebraic lookup
\newcommand{\relalookup}{\REL_{\p{AlLook}}}
\newcommand{\relalookupeq}{\REL_{\p{EqAlLook}}}
\newcommand{\relaindexedlookup}{\REL_{\p{AlIdxLook}}}
\newcommand{\relccl}{\REL_{\p{ConstAlCmLook}}}
%%committed lookup relations
\newcommand{\relMlookupcom}{\REL_{\p{CmMAlLook}}}

\newcommand{\cm}{\p{cm}}

\newcommand{\meq}{\multilinmle{\p{eq}}}
\newcommand{\miota}{\multilinmle{\iota}}
\newcommand{\mtau}{\multilinmle{\tau}}
\newcommand{\Racc}{\mathcal{R}_{\mathrm{acc}}}
\newcommand{\Relax}{\mathcal{R}_{\mathrm{rR1CS}}}
\newcommand{\Reval}{\mathcal{R}_{\mathrm{eval}}}
\newcommand{\Rsmol}{\mathcal{R}_{\mathrm{decR1CS}}^b}
\newcommand{\Fp}{\mathbb{F}_p}
\newcommand{\nin}{n_{in}}
\newcommand{\R}{\mathcal{R}}
\newcommand{\Rp}{\mathcal{R}_p}
\newcommand{\C}{\mathcal{C}}
\newcommand{\assgn}{\xleftarrow{}}
\newcommand{\flatten}[1]{#1^\flat}
\newcommand{\deflatten}[1]{#1^\sharp}

\renewcommand{\norm}[1]{\left\lVert#1\right\rVert_{\infty}}
\newcommand{\norminf}[1]{\norm{#1}_\infty}
\newcommand{\normop}[1]{\norm{#1}_{\mathrm{op}}}
\newcommand{\Ajt}{\mathcal{A}jt}
\newcommand{\NTT}{\p{NTT}}
\newcommand{\RotSum}{\p{RotSum}}
\newcommand{\bsmol}{(1,\:b,\ldots,\:b^{k-1})}
\newcommand{\vvec}{\p{vec}}
\newcommand{\draw}{\xleftarrow{\$}}
\newcommand{\Csmall}{\mathcal{C}_{\mathrm{small}}}
\newcommand{\stark}{2^{251}+17\cdot2^{192}+1}
\newcommand{\circlep}{2^{31}-1}
\newcommand{\babybear}{15 \cdot 2^{27} + 1}
\newcommand{\goldilocks}{2^{64} - 2^{32}  + 1}

\newcommand{\ronecs}{\p{R1CS}}
\newcommand{\cronecs}{\p{cR1CS}}
\newcommand{\crronecs}{\p{crR1CS}}
\newcommand{\red}{\p{red}}
\newcommand{\iop}{\p{IOP}}
\newcommand{\I}{\mathcal{I}}

\newcommand{\Mod}[1]{\ (\mathrm{mod}\ #1)}
\newcommand{\ord}{\mathrm{ord}}
\newcommand{\N}{\mathbb{N}}
\newcommand{\Z}{\mathbb{Z}}
\newcommand{\inter}{\langle \Prover^*, \Verifier \rangle}
\newcommand{\hinter}{\langle \Prover, \Verifier \rangle}
%To have widebar command without adding the whole package
\DeclareFontFamily{U}{mathx}{\hyphenchar\font45}
\DeclareFontShape{U}{mathx}{m}{n}{
      <5> <6> <7> <8> <9> <10>
      <10.95> <12> <14.4> <17.28> <20.74> <24.88>
      mathx10
      }{}
\DeclareSymbolFont{mathx}{U}{mathx}{m}{n}
\DeclareFontSubstitution{U}{mathx}{m}{n}
\DeclareMathAccent{\widebar}{0}{mathx}{"73}



\newcommand{\blue}[1]{\textcolor{blue}{#1}}
\newcommand{\lin}{\mathcal{L}}

\newcommand{\Requal}{\REL_{\mathsf{equal}}}


\newcommand{\Prover}{\prover}
\newcommand{\Verifier}{\verifier}


\newcommand{\relsumcheck}{\REL_{\p{sc}}}
\newcommand{\releval}{\REL_{\p{eval}}}


\newcommand{\Rccsnew}{\REL_{\mathsf{cR1CS}}}
\newcommand{\Rrccs}{\REL_{\mathsf{rcR1CS}}}
\newcommand{\Rrcs}{\REL_{\mathsf{rR1CS}}}

\newcommand{\gen}{\mathsf{Gen}}
\newcommand{\pparams}{\mathsf{gp}}


    \newcommand{\eabort}{\mathcal{E}_{\p{abort}}}
        \newcommand{\evalidwit}{\mathcal{E}_{\p{valid-wit}}}
    \newcommand{\pro}{\p{Pr}}
    \newcommand{\expectation}{\mathbb{E}}
    \newcommand{\cE}{\mathcal{E}}
            \newcommand{\cElines}{\cE_{\mathsf{lines}}}

                \newcommand{\eone}{\mathcal{E}_{\p{root}}^{\EE^{(1)}}}
        \newcommand{\etwo}{\mathcal{E}_{\p{good-run}}}


            \newcommand{\UU}{\mathbf{U}}
    \newcommand{\VV}{\mathbf{V}}
		\newcommand{\cEzero}{\cE_{\p{zero}}}
		\newcommand{\cEnonzero}{\cE_{\p{nonzero}}}
		\newcommand{\cEbinding}{\cE_{\p{binding}}}

  \newcommand{\cEbadlines}{\mathcal{E}_{\p{bad-lines}}}


  \renewcommand{\gp}{\mathsf{gp}}
\newcommand{\vp}{\mathsf{vp}}
\newcommand{\idxer}{\mathcal{I}}

\renewcommand{\oracle}[1]{[[{#1}]]}
\newcommand{\loc}[1]{\ZZ_{(#1)}}
\newcommand{\vv}{\mathbf{v}}
\newcommand{\cC}{\mathcal{C}}
\newcommand{\cR}{\mathcal{R}}

\newcommand{\cL}{\mathcal{L}}
\newcommand{\lift}{\pcvarstyle{lift}}
\newcommand{\bounds}{\ppp}


\newcommand{\cQ}{\mathcal{Q}}
\newcommand{\cF}{\mathcal{F}}
\newcommand{\soundness}{\mathsf{sound}}
\newcommand{\boundedsoundness}{\mathsf{bound-sound}}

\newcommand{\completeness}{\mathsf{comp}}
%\renewcommand{\multilin}{\mathsf{multilin}}

\newcommand{\nummorph}{\eta}
\newcommand{\cA}{\mathcal{A}}
\newcommand{\valid}{\mathsf{wf}}
\newcommand{\Enc}{\mathsf{Enc}}

\newcommand{\codelength}{\mathsf{n}}
\newcommand{\codedim}{\mathsf{dim}}
\newcommand{\uu}{\mathbf{u}}
\newcommand{\T}{\mathsf{T}}
\newcommand{\bdsamplingset}{S}
\newcommand{\cP}{\mathcal{P}}
\renewcommand{\qq}{\mathbf{q}}
\newcommand{\local}[1]{\ZZ_{(#1)}}
\newcommand{\bounded}{\mathsf{bounded}}
\newcommand{\rate}{\rho}
\newcommand{\test}{\mathsf{test}}
\newcommand{\distance}{\mathsf{dist}}
\newcommand{\dd}{\mathbf{d}}
\newcommand{\accept}{\mathsf{accept}}
\newcommand{\bdpoly}{\mathsf{bdpoly}}
\renewcommand{\aa}{\mathbf{a}}
\newcommand{\Runtime}{\mathsf{Runtime}}
\newcommand{\Encbinary}{\mathsf{Enc}_{\mathsf{binary}}}
\newcommand{\wt}{\mathsf{wt}}
\newcommand{\supp}{\mathsf{supp}}

\newcommand{\Ber}{\mathsf{Ber}}
\newcommand{\BP}{\mathsf{BP}}
\newcommand{\rk}{\mathsf{rk}}
\newcommand{\sgn}{\mathsf{sgn}}

    \newcommand{\PC}{\mathsf{PC}}
    \newcommand{\LA}{\mathsf{LA}}
    \newcommand{\prove}{\mathsf{Prove}}


    \newcommand{\advA}{\mathcal{A}}
    \newcommand{\Commit}{\mathsf{Commit}}
    \newcommand{\Open}{\mathsf{Open}}
    \newcommand{\Eval}{\mathsf{Eval}}
    \newcommand{\binding}{\mathsf{binding}}



\DeclareMathOperator{\Frac}{Frac}
\newcommand{\lcm}{\text{lcm}}

\newcommand{\Ext}{\mathsf{Ext}}

\newcommand{\record}{\mathsf{Record}}
%\newcommand{\ks}{\mathsf{ks}}
\newcommand{\ww}{\mathbf{w}}
%  \renewcommand{\Span}{{\mathsf{span}}}
\newcommand{\li}{\mathsf{l.i.}}
\newcommand{\ld}{\mathsf{l.d.}}
\newcommand{\boundgenerator}{\lVert M_\cC \rVert_\infty}
\newcommand{\CCS}{\mathsf{CCS}}
\newcommand{\RICS}{\mathsf{R1CS}}
\newcommand{\ZZZ}{\mathbf{Z}}

   % Define macros for the procedures
    \newcommand{\Setup}{\mathsf{Setup}}
    \newcommand{\Com}{\mathsf{Com}}
    \newcommand{\ProveEvalMod}{\mathsf{ProveEvalMod}}
    \newcommand{\VfyEvalMod}{\mathsf{VfyEvalMod}}
    
    % Define macros for common symbols
    \newcommand{\opn}{\mathsf{opn}}
    \newcommand{\cmt}{\mathsf{cm}}

\newcommand{\BDtestsamplingprime}{{q_0}}
\newcommand{\codelin}{{\cC_{\lambda}}}
\newcommand{\boundq}{{B_{\mathsf{pt}}}}
\newcommand{\boundp}{\boundq}

\newcommand{\boundv}{{B_{\vv}}}

\newcommand{\primeset}{{\cP_{\lambda}}}
\newcommand{\proj}{{\mathsf{proj}}}
\newcommand{\total}{\mathsf{total}}

\newcommand{\extraction}{\mathsf{extraction}}
\renewcommand{\time}{\mathsf{time}}

\newcommand{\codelinplain}{\cC}
\newcommand{\good}{\mathsf{good}}
\newcommand{\len}{\mathsf{len}}
\newcommand{\witnessbound}{{2\cdot\boundv + (6\codedim+2)\cdot \log(\BDtestsamplingprime\cdot \codedim)}}
%\newcommand{\witnessbound}{{\boundv + \log(\BDtestsamplingprime^3\cdot \codedim)}}
\newcommand{\witnessboundencoded}{\log(\norm{M_\codelin}\cdot \codedim) +\witnessbound }

\newcommand{\pos}{\mathsf{pos}}
\renewcommand{\LANG}{\mathsf{LANG}}

\newcommand{\cM}{\mathcal{M}}
\newcommand{\relevalppp}{(\mu, \boundv, \boundp, \delta, \BDtestsamplingprime)}

\newcommand{\words}{\mathsf{Words}}

\newcommand{\thetaexpression}{\frac{2\cdot \codedim^2\cdot \left( \boundv +\log(\codedim)  + 2 \cdot \log(\BDtestsamplingprime)\right) + \boundv +\mu\cdot \boundp  + 3\mu+2}{\lambda\cdot |\primeset|}}

\newcommand{\thetaexpressionnotreduced}{\frac{ 2\cdot\codedim^2\cdot \left( \log(\norm{\words(\hat{\uu}})) + 2 \cdot \log(\BDtestsamplingprime)\right) + \boundv +\mu\cdot \boundp  + 3\mu+2}{\lambda\cdot |\primeset|}}

\newcommand{\lo}{{\mathsf{lo}}}

\newcommand{\tot}{{\mathsf{to}}}

\newcommand{\ex}{{\mathsf{ex}}}
%\newcommand{\size}{{\mathsf{size}}}
\newcommand{\mul}{{\mathsf{mul}}}
\newcommand{\samp}{{\mathsf{sample}}}
\newcommand{\jea}{{\mathsf{JEA}}}
\newcommand{\ns}{{\mathsf{ns}}}

\newcommand{\cD}{\mathcal{D}}
\renewcommand{\tt}{\mathbf{t}}
\newcommand{\Zinc}{\mathsf{Zinc}}
\newcommand{\Zip}{\mathsf{Zip}}
\newcommand{\zipp}{\textsf{Zip+}\xspace}



\newcommand{\PIOP}{\mathsf{PIOP}}
\newcommand{\IOP}{\mathsf{IOP}}

\newcommand{\inproof}{\mathsf{IP}}
\newcommand{\indexer}{\mathsf{Indexer}}
\newcommand{\vparams}{\mathsf{vp}}
\newcommand{\gparams}{\mathsf{gp}}

\newcommand{\bA}{\mathbf{A}}
\newcommand{\bB}{\mathbf{B}}
\newcommand{\bC}{\mathbf{C}}

\renewcommand{\pp}{\mathsf{pp}}

\newcommand{\hint}{\mathsf{hint}}

\newcommand{\weirdrationals}{\QQ}%$[\hat{\uu}]}

\newcommand{\Irr}{\mathsf{Big}}

 \newcommand{\symbolbigbound}{\mathsf{BoundEnc}}

 \newcommand{\compiled}{\mathsf{compiled}}
 \newcommand{\succinct}{\mathsf{succ}}
 \newcommand{\vcom}{\mathsf{VC}}
 \newcommand{\JEA}{\mathsf{JEA}}
  \newcommand{\weight}{\mathsf{weight}}
  \newc{\dom}{\mathcal{D}}

%%%%%%%%%%%%%%%%%%%%%%%%%%%%%%%%%%%%%%%%%%%%%%%%%%%%%%%%%%%%%%%%%%%%%%%%%%%%%%%%
\title{\stylizedtitle}

\newcommand{\FormatAuthor}[3]{%
\begin{tabular}{c}
#1 \\ {\small\texttt{#2}} \\ {\small #3}
\end{tabular}
}

\author{
	Albert Garreta, Psi Vesely, Arantxa Zapico
}
\date{\today}

%%%%%%%%%%%%%%%%%%%%%%%%%%%%%%%%%%%%%%%%%%%%%%%%%%%%%%%%%%%%%%%%%%%%%%%%%%%%%%%%
\begin{document}
\maketitle

\begin{abstract}
\end{abstract}

\setcounter{tocdepth}{2}
\begin{spacing}{0.8}
{\footnotesize \tableofcontents}
\end{spacing}
%%%%%%%%%%%%%%%%%%%%%%%%%%%%%%%%%%%%%%%%%%%%%%%%%%%%%%%%%%%%%%%%%%%%%%%%%%%%%%%%
%%%%%%%%%%%%%%%%%%%%%%%%%%%%%%%%%%%%%%%%%%%%%%%%%%%%%%%%%%%%%%%%%%%%%%%%%%%%%%%%
%%%%%%%%%%%%%%%%%%%%%%%%%%%%%%%%%%%%%%%%%%%%%%%%%%%%%%%%%%%%%%%%%%%%%%%%%%%%%%%%
\section{Preliminaries}


\subsection{Rings and rings of polynomials} Throughout the paper we extensively work with multivariate polynomials whose coefficients are polynomials themselves. Formally, given a ring $\cR$ and variables $\YY=(Y_1,\ldots, Y_\mu)$, we let $\cR[\YY]$ denote the ring of multivariate polynomials on variables $\YY$, with coefficients inn $\cR$. For us, typically $\cR$ will be of the form $\ZZ[X]$ or $\QQ[X]$ (or even $\local{q}[X]$ for some prime $q$). In that case, $\cR[\YY]$ contains all polynomials on variables $\YY$ whose coefficients are polynomials from $\ZZ[X]$ (or $\QQ[X]$, or $\local{q}[X]$). 

The ring $(\ZZ[X])[\YY]$ is isomorphic to the ring $\ZZ[X, \YY]$ of all polynomials with integer coefficients on variables $(X,\YY)$  (and similarly for $(\QQ[X])[\YY]$ and $(\local{q}[X])[\YY]$). However, in this work we do not use this isomorphic representation of our polynomials, and insist on viewing elements from $(\ZZ[X])[\YY]$ as polynomials on the variables $\YY$ with coefficients in $\ZZ[X]$. We denote such a polynomial by $$f(X;\YY),$$
highlighting the presence of the variable $X$. The reason for doing so is that sometimes we will want to evaluate $X$ at a specific value $x$, obtaining a polynomial $f(x;\YY)$ in $\ZZ[\YY]$, and sometimes we will need to evaluate the variables $\YY$ at a specific values $\yy$, and we will look at the result as  a polynomial $f(X,\yy)$ in $\ZZ[X]$. 

In general, we denote vectors of elements with lowercase boldface letters, e.g.\ $\vv,\uu,$ etc. Given a ring $\cD$ (typically $\ZZ, \QQ$, or $\local{q}$), we denote vectors of elements from $(\cD[X])[\YY]$ as $\vv(X;\YY), \uu(X;\YY)$, etc.


For every ring $\cR$ we work with we fix a publicly known representation of its elements as strings of bits. By $\cR_{\leq B}$ we denote the subset of $\cR$ formed by all elements whose bitstring representation contains at most $B$ bits. In \cref{s: bistring_reps} we describe the representation we use for the rings $\ZZ,\QQ, \ZZ[X],$ and $\QQ[X]$ \albert{And $\local{q},\local{q}[X]$?}.

\subsection{Bitstring representation of ring elements}\label{s: bistring_reps}
\albert{To do.}

\section{Algebraic constraints over $\QQ[X]$ with evaluation predicates}

In this section we define the type of constraints we are interested in working on. As we argued in \cref{?}, and as we see later in \cref{?}, these constraints are highly expressive, and can  express essentially all computations of interest with almost no arithmetization overhead.


Let $\cQ$ be a set of polynomials with coefficients in $\QQ[X]$ (possibly multivariate and of arbitrary degree). Let $\gp = (k, m, n, \mu, B)$ be global parameters, where $k,m,n,B$ are size parameters. An algebraic relation for $(\gp,\cQ)$ is a set $\REL_{\gp, \cQ}$ of triples $(\idx, \inp;\wit)$ with the following properties:
\begin{itemize}
	\item The \emph{index} $\idx$ contains $n$ oracles $\oracle{g_1(X;\YY)}, \ldots, \oracle{g_n(X;\YY)}$ to multilinear polynomials $(\QQ[X])_{\leq B}^\multilin[\YY]$, where $\YY=(Y_1,\ldots, Y_\mu) $ is a tuple of variables.
	 \item $\cQ$ is a set of polynomials with coefficients in $\QQ[X]$, each on $(n+ k)\cdot 2^\mu + m$ variables.
	\item $\wit$ is a vector consisting of $k$ multilinear polynomials $f_1(X;\YY),\ldots, f_k(X;\YY)$ from $(\QQ[X])_{\leq B}^{\multilin}[\YY]$. %and the multilinear polynomials $g_1,\ldots, g_n\in \R_B^\multilin[\XX]$.
	\item $\inp=(\zz(X), \oracle{f_1(X;\YY)}, \ldots, \oracle{f_k(X;\YY))}$, where $\zz\in (\QQ[X])_{\leq B}^m$.
	\item Each of the polynomials in $\cQ$ vanishes (as a polynomial in $\QQ[X]$) when evaluated on the values $$((g_1(X;\yy),\ldots, g_n(X;\yy), f_1(\yy),\ldots, f_k(X;\yy))_{\yy\in \BB^\mu}, \yy(X)).$$ 
	\end{itemize}

\begin{equation*}
\begin{aligned}
	
\end{aligned}	
\end{equation*}

    \subsection{PIOP over a ring, projections, and lifts}

    We start by outlining our setting and presenting some tools we will work with. Mainly, we define what we mean by a relation to be algebraic over a ring $\R$, and we establish correspondences between PIOPs for a relation over a ring $\R$, and a PIOP for an homomorphic image $\R'$ of $\R$. A reader wishing to tone down the abstractness of the presentation can replace $\R$ by the local ring $\local{q}$ (or even $\ZZ$, even though we will never instantiate the schemes in this section on $\ZZ$), and $\R'$ by $\FF_q$, for $q$ a prime.

    
    Recall that, given a ring $\R$, we implicitly fix an encoding of its elements as strings of bits. Given $B\geq 1$, by $\R_B$ we denote the subset of $\R$ consisting of ring elements whose encoding has at most $B$ bits.

    \begin{definition}[Algebraic  indexed relation over a ring $\R$]
    \label{d: Algebraic  Indexed Relation over a ring}
    Let $\R$ be a ring, and let $\cQ$ be a set of polynomials with coefficients in $\R$ (possibly multivariate and of arbitraty degree). Let $\gparams=(k,m,n,\mu, B)$ be  global parameters where $k,m,n,\mu,B$ are size parameters.  Abusing the language, we set $\gparams$ to also include the security parameter $\lambda$, the ring $\R$, and the polynomials $\cQ$, but we do not explicitly display them inside $\gparams$. Instead, we refer to $\R, \cQ$ in more explicit ways. We do so because in our constructions, $\gparams$ stays fixed, while $\R$ and $\cQ$ often vary.
    
    An \emph{algebraic indexed relation for the parameters $(\gparams, \R, \cQ)$}  is a set $\REL_{\gparams,\R,\cQ}$ of triples $(\idx, \inp; \wit)$ with the following properties:
    
        \begin{itemize}
            \item The \emph{index} $\idx$ contains  $n$ oracles $\oracle{g_1},\ldots, \oracle{g_n}$ to multilinear polynomials $g_1,\ldots, g_n\in \R_B^\multilin[\XX]$, where $\XX=(X_1,\ldots, X_\mu)$.
            \item $\cQ$ is a set of polynomials with coefficients in $\R$, each on $(n+ k)\cdot 2^\mu + m$ variables.
            \item $\wit$ is a vector consisting of $k$ multilinear polynomials $f_1(\XX),\ldots, f_k(\XX)$ from $\R_B^{\multilin}[\XX]$. %and the multilinear polynomials $g_1,\ldots, g_n\in \R_B^\multilin[\XX]$.
            \item $\inp=(\yy, \oracle{f_1}, \ldots, \oracle{f_k})$, where $\yy\in \R_B^m$.
            \item Each of the polynomials in $\cQ$ vanishes when evaluated on the values $$((g_1(\xx),\ldots, g_n(\xx), f_1(\xx),\ldots, f_k(\xx))_{\xx\in \BB^\mu}, \yy).$$ 
        \end{itemize}
    
    Formally, $\REL_{\gparams,\R,\cQ}$ has the following form:
    %
    \begin{equation*}
    \begin{aligned}
    \REL_{\gparams,\R,\cQ} = \left\{ (\idx, \inp ; \wit) \left| \ \begin{aligned}
    & \gparams=(k,m,n,\mu, B),  \\
    & \idx = ( \oracle{g_1},\ldots, \oracle{g_n}), \\
    & \inp =(\yy, \oracle{f_1}, \ldots, \oracle{f_k}) \text{ for some } \yy\in \R_B^{m},\\    
        &\wit = (f_1(\XX), \ldots, f_k(\XX)) \in \left(\R_B^{^{\multilin}}[\XX]\right)^{n},    \ \XX=(X_1,\ldots, X_\mu),\\
        &(g_1(\XX),\dots,g_n(\XX))\in \left(\R_B^{^{\multilin}}[\XX]\right)^{n},\\
        &Q((g_1(\xx),\ldots, g_n(\xx), f_1(\xx),\ldots, f_k(\xx))_{\xx\in \BB^\mu}, \yy) =0  \text{ for all } Q\in \cQ
    \end{aligned} \right.\right\}
    \end{aligned}
    \end{equation*}
    \end{definition}
  %
    In \cref{ex: The CCS relation as an algebraic indexed relation} we show that both the CCS relation and the lookup relation (\cref{s: CCS}) can be rewritten as algebraic indexed relations.  In the next definition we generalize the notion of algebraic indexed relation.
%
      \begin{definition} [Projected algebraic indexed relation]\label{d: Projected Algebraic  Indexed Relation over a ring}
    Let $\REL_{\gparams,\R,\cQ}$ be an algebraic indexed relation with parameters $(\gparams, \R, \cQ)$, and let $\phi:\R \to \R'$ be a ring homomorphism. We define an associated relation, which we call \emph{$\phi$-projected algebraic indexed relation}, denoted by $\phi(\REL_{\gparams,\R, \cQ})$, as the relation $\REL_{\gparams,\R, \cQ}$, with the only difference that in $\phi(\REL_{\gparams,\R, \cQ})$ we require that the image by $\phi$ of the polynomials in $\cQ$ vanishes. Formally (we highlight the difference between $\REL_{\gparams,\R, \cQ}$ and $\phi(\REL_{\gparams,\R, \cQ})$ in blue):
    %
    \begin{equation*}
    \begin{aligned}
    \phi(\REL_{\gparams,\R,\cQ}) = \left\{ (\idx, \inp ; \wit)\left| \ \begin{aligned}
    &\gparams \coloneqq(k,m,n,\mu, B),\  \\
    &\idx \coloneqq(\oracle{g_1},\ldots, \oracle{g_n}), \\
    &\inp =(\yy, \oracle{f_1}, \ldots, \oracle{f_k}) \text{ for some } \yy\in \R_B^{m},\\
        &\wit = (f_1(\XX), \ldots, f_k(\XX)) \in \left(\R_B^{^{\multilin}}[\XX]\right)^{k},    \ \XX=(X_1,\ldots, X_\mu),\\
        &(g_1(\XX),\dots,g_n(\XX))\in \left(\R_B^{^{\multilin}}[\XX]\right)^{n},\\
        &\textcolor{blue}{\phi(Q((g_1(\xx),\ldots, g_n(\xx), f_1(\xx),\ldots, f_k(\xx))_{\xx\in \BB^\mu}, \yy)) =0} \\ &\quad\quad\quad\quad\quad\quad\quad \quad\quad\quad\quad\quad\quad\quad\quad\quad\quad\quad\text{ for all } Q\in \cQ 
    \end{aligned} \right.\right\}
    \end{aligned}
    \end{equation*}
    \end{definition}
    Importantly, in a projected algebraic indexed relation, the oracles and polynomials in $(\idx,\inp;\wit)$ are over the ring $\R$, even though the constraints posed by the polynomials $\cQ$ are enforced only under the image of $\phi$. 

    Notice that, by taking $\R'=\R$ and $\phi$ the identity homomorphism (i.e.\ $\phi(a)=a$ for all $a\in \R$) we have that $\phi(\REL_{\gparams,\R, \cQ}) = \REL_{\gparams,\R, \cQ}$. Hence, projected algebraic indexed relations are a generalization of algebraic indexed relations. 

    \begin{definition}[Well-formed index-instance pairs]\label{d: valid_index_instance_pair}
        Let $\phi(\REL_{\gparams,\R, \cQ})$ be a projected algebraic indexed relation. Following \cref{r: well_formed_convention}, we say that $(\idx,\inp)$ is a \emph{well-formed index-instance pair for the relation $\phi(\REL_{\gparams,\R, \cQ})$} if $\idx$ and $\inp$ have the form specified in the definition of $\phi(\REL_{\gparams,\R, \cQ})$. Namely, if $\idx=(\gparams,\oracle{g_1},\ldots, \oracle{g_n})$ and $\inp= (\yy, \oracle{f_1},\ldots, \oracle{f_k})$  where $g_1, \ldots, g_n, f_1,\ldots, f_k$ are all $\mu$-variate multilinear polynomials from $\R_B[\XX]$, and $\yy$ is a tuple of $m$ elements from $\R_B$.
        %\hendrik{Now that we have that here, do we still want to keep it in the preliminaries?}
    \end{definition}

\begin{comment}
   \begin{definition}[{Algebraic PIOP for a  (projected) algebraic indexed relation over $(\R,\cQ)$}]
   
   Let $\REL_{\R,\cQ}$ be an algebraic indexed relation over $(\R,\cQ)$, and let $\phi : \R \to \R'$ be a ring homomorphism (possibly the identity homomorphism). %Let $\cF$ be a map assigning tuples of natural numbers to tuples of polynomials over $\R'$.  \albert{We are not really using the $\cF$ polynomials.}
   
   An \emph{Algebraic PIOP} $\prot$ for $\phi(\REL_{\R,\cQ})$ and $\cF$ is a PIOP for $\phi(\REL_{\R,\cQ})$ in which $\verifier$ is given as input a well-formed  index-instance pair $(\idx, \inp)$ for the relation  $\phi(\REL_{\R,\cQ})$, and $\prover$ is given additionally a witness $\wit$ such that $(\idx, \inp;\wit) \in \phi(\REL_{\R,\cQ})$. In this PIOP, all oracles sent by the prover are oracles to multilinear %\albert{ok or too restrictive?} \hendrik{Should be fine given the MLE, no?}  \albert{I mean that we could allow prover to send polynomials of arbitrary degrees} 
   polynomials with coefficients in $\R'$, and all messages from $\verifier$  are uniformly sampled in finite subsets of $\R'$. %Further, at the end of the interaction with the prover,  after having queried the received oracles, $\verifier$ accepts or rejects based on whether the polynomials in $\cF(\gparams)$ are zero when evaluated on the verifier's queried values and ring elements in $\inp$ (after applying $\phi$ to them if they belong to $\R$), and the messages sent by $\verifier$.   
   \albert{Is this formal enough?}\hendrik{I think it is fine, I think we should have to the same degree of formality as the initial PIOP definition.}

   \end{definition}
\end{comment}
     %
     %Formally, $\prot=(\prover, \verifier)$ is a PIOP for $\REL_{\R, S, \cQ}$.  At round $i\in[k-1]$, $\prover$ sends an oracle $\oracle{h_i}$ to $\verifier$, where $h_i\in \R^{\leq d_i}[X_1,\ldots, X_{n_i}]$ and $d_i, n_i\geq 1$ are parameters specified in $\gp \gets \setup(1^\lambda)$. $\verifier$ replies with a vector of ring elements uniformly sampled over a finite prescribed subset $\codelin_i$ of $\R^{t_i}$, where $\codelin_i$ and $t_i$ are again specified in $\gp$. At the last round, $\prover$ sends one last oracle $\oracle{f_k}$, with $h_k\in \RR^{\leq d_k}[X_1,\ldots, X_{n_k}]$. $\verifier$ then makes  queries to the oracles $\oracle{h_1},\ldots, \oracle{h_k}$ at  random points, say it queries $\oracle{h_i}$ at points $\xx_{i1},\ldots, \xx_{im_i}\in \RR^{n_i}$, for each $i\in [k]$, where the $m_i$ are parameters specified in $\gp$ \albert{It can also query the oracles in the index and in the instance}.  Let $\yy = (y_{ij})_{i\in [k],\ j\in [m_i]}$ be the values received from querying the oracles. Then  $\verifier$ checks whether \albert{we should also allow $\verifier$ to use the ring elements in $\inp'$}
     %
     %$$
    %F(\yy) = 0,
     %$$
     %where $F$ is a polynomial from $\R[Y_1,\ldots, Y_{\sum_{i\in [k]} m_i}]$ which is specified in $\gp$. Then $\verifier$ accepts (returns $1$) or rejects (returns $0$) depending on whether $F(\yy) = 0$ or not.
    

    
    % \begin{definition} [Fully projected algebraic indexed relation]
    % Let $\REL_\R$ be an algebraic indexed relation over $(\R,S)$, and let $\phi:\R \to \R'$ be a ring homomorphism.
    % We define an associated relation called \emph{$\phi$-fully projected algebraic indexed relation} as the relation $\REL_{\phi(\R), \phi(S)}$. Formally:
    % %
    %     \begin{equation*}
    % \REL_{\phi(\R), \phi(S)}) = \left\{ (\idx, \inp;\wit) \left| \ \begin{aligned}&\text{There exists } (\idx_\R, \inp_\R;\wit_\R)\in \REL_{\R,\cQ} \text{ such that: } \\ &\inp=\phi(\inp_\R),\ \wit = \phi(\wit_\R),\\
    % & \idx_\R = (k,m,n,\mu, \oracle{g_1},\ldots, \oracle{g_n}, Q),\\
    % &\idx = (k,m,n,\mu, \oracle{\phi(g_1)}, \ldots, \oracle{\phi(g_n)}, \phi(Q))\end{aligned} \right.\right\}
    % \end{equation*}
    % \end{definition}

    Let $\phi:\R\to \R'$ be a ring homomorphism and let $\REL_{\gparams,\R, \cQ}$ be an algebraic indexed relation. By $\phi(\cQ)$ we define the set containing the image under $\phi$ of the polynomials in $\cQ$. One can then consider the relation $\REL_{\gparams,\R', \phi(\cQ)}$. By definition, this is precisely, 
    %
    \begin{equation*}
    \begin{aligned}
    \REL_{\gparams,\R',\phi(\cQ)} = \left\{ (\idx, \inp ; \wit)\left| \ \begin{aligned}
    &\gparams = (k,m,n,\mu, B),\ n,m,k,B\geq 1, \\
    &\idx = (\oracle{g_1},\ldots, \oracle{g_n}), \\
    &\inp =(\yy, \oracle{f_1}, \ldots, \oracle{f_k}) \text{ for some } \yy\in \left(\R_B'\right)^{m},\\
        &\wit = (f_1(\XX), \ldots, f_k(\XX)) \in \left(\R_B'{}^{^{\multilin}}[\XX]\right)^{k}, \  \XX=(X_1,\ldots, X_\mu),\\
        &(g_1(\XX),\dots,g_n(\XX)\in \left(\R_B'{}^{^{\multilin}}[\XX]\right)^{n},\\
        &\phi(Q)((g_1(\xx),\ldots, g_n(\xx), f_1(\xx),\ldots, f_k(\xx))_{\xx\in \BB^\mu}, \yy) =0 \\ &\quad\quad\quad\quad\quad\quad\quad \quad\quad\quad\quad\quad\quad\quad\quad\quad\quad\quad\text{ for all } Q\in \phi(\cQ)
    \end{aligned} \right.\right\}
    \end{aligned}
    \end{equation*}

    The following definition is necessary to make sure many of the constructions in this section are well defined and result in efficient algorithms. 

    \begin{definition}[Efficient homomorphism $\phi$]\label{d: efficient_homomorphism}
        We say that a ring homomorphism $\phi:\R \to \R'$ is \emph{efficient} if 1) $\phi(\R_B)= \R_B'$\footnote{We ask that  bit-size bound in $\R$ and in $\R'$ is the same simply due to ease of presentation. The  condition could be relaxed to asking that $\phi(\R_B)= \R_{B'}'$ for a new parameter $B'$. In that case, our results still hold after making straightforward changes to their statements and corresponding proofs.} for all $B\geq 1$ (in words, the image by $\phi$ of the elements from $\R$ of bit-size less than $B$ can be written with less than $B$ bits); 2) $\phi(a)$ can be computed in polynomial time on the bit-size of $a$, for all $a\in \R$; and 3) given $a'\in \R_B'$, it is possible to find $a\in \R_B$ such that $\phi(a)=a'$ in polynomial time on the bit-size of $a'$.
    \end{definition}

    Given an index $\idx=(\oracle{g_1},\ldots, \oracle{g_n})$ for $\REL_{\gparams, \R, \cQ}$ and a homomorphism $\phi:\R \to \R'$, we define $\phi(\idx)=( \phi(\oracle{g_1}),\ldots, \phi(\oracle{g_n}))$.

The next observation follows immediately from \cref{d: efficient_homomorphism}.
\begin{remark}\label{r: well_formedness_preservation}
Suppose $\phi:\R \to \R'$  is an efficient ring homomorphism. Then, given any well-formed index-instance pair $(\idx,\inp)$ for $\phi(\REL_{\gparams,\R,\cQ})$, we have that $(\phi(\idx), \phi(\inp))$ is a well-formed index-instance pair for $\REL_{\gparams,\R', \phi(\cQ)}$ which can be computed efficiently. 

Conversely, given a well-formed index-instance pair $(\idx',\inp')$ for $\REL_{\gparams,\R',\cQ'}$, there exists a well-formed index-instance pair $(\idx,\inp)$ for $\phi(\REL_{\gparams,\R,\cQ})$ that can be computed efficiently.
\end{remark}

    
    Given an oracle $\oracle{f}$ to a polynomial $f\in \R[\XX]$, we let $\phi(\oracle{f})= \oracle{\phi(f)}$. In the scenario where $\verifier$ has received an oracle $\oracle{f}$, $\verifier$ can query the oracle $\phi(\oracle{f})$ as follows: first, it queries $\oracle{f}$ at the desired position, and then $\verifier$ applies the homomorphism $\phi$ to the received value. %\albert{I guess we have to say somewhere that $\phi$ is known to everyone and efficiently computable}\hendrik{For the former, can we just make it part of the statement?}.

   \begin{lemma}\label{l: valid_instance_preserved_by_phi}
        Let $\phi:\R\to \R'$ be an efficient ring homomorphism, and    let  $(\idx,\inp)$ be a well-formed index-instance pair for  $\phi(\REL_{\gparams,\R, \cQ})$. Then   $(\idx,\inp)\in \LANG(\phi(\REL_{\gparams,\R, \cQ}))$ if and only if $(\phi(\idx),\phi(\inp))\in \LANG(\REL_{\gparams,\R', \phi(\cQ)})$. 
   \end{lemma}

  
     Now, let $\prot_\R'$ be a PIOP over $\R'$ for $\REL_{\gparams,\R', \phi(\cQ)}$. We introduce the notion of the lift of $\prot_\R'$. Informally, this is a PIOP over $\R$ for $\phi(\REL_{\gparams,\R, \cQ})$ where $\prover$ and $\verifier$ simply apply the map $\phi$ to all  elements from $\R$ and polynomials with coefficients in $\R$, and then execute $\prot_\R'$.

        \begin{definition}[Lift of a PIOP] \label{d: Lift of an Algebraic PIOP}
     Let $\REL_{\gparams,\R,\cQ}$ be an algebraic indexed relation with parameters $(\gparams, \R,\cQ)$. Let $\phi:\R \to \R'$ be an efficient (cf.\ \cref{d: efficient_homomorphism}) ring homomorphism.  Let $\prot_{\R'}=( \indexer_{\R'}, \prover_{\R'},\verifier_{\R'})$ be a PIOP over $\R'$ for $\REL_{\gparams,\R',\phi(\cQ)}$. In \cref{a: lift} we describe a PIOP over $\R$ for $\phi(\REL_{\gparams,\R, \cQ})$, denoted $\prot_{\R'}^{\lift}= (\indexer_{\R'}^\lift, \prover_{\R'}^\lift,\verifier_{\R'}^\lift)$, and called the \emph{lift of $\prot_\R'$ onto $\R$}. 

        	\begin{algorithm}[H]
		\caption{A PIOP $\prot_{\R'}^{\lift}=(\indexer_{\R'}^\lift,\prover_{\R'}^\lift, \verifier_{\R'}^\lift)$ over $\R$ for the relation $ \phi(\REL_{\gparams, \R,\cQ})$, called the \emph{lift} of $\prot_\R'=(\indexer_{\R'},\prover_{\R'}, \verifier_{\R'})$. \label{a: lift}}%\begin{algorithmic}[H]
		%{}
		            
        \textbf{Indexer:}      Given $\gparams$ and $\idx=(\oracle{g_1},\ldots, \oracle{g_n})$ for $\phi(\REL_{\gparams,\R, \cQ})$, the indexer $\indexer_{\R'}^\lift$ runs $\indexer_{\R'}$ on input $\gparams$ and $\phi(\idx)$, and obtains verifier and prover parameters $\vp, \pp$ as output. By definition (cf.\ \cref{s: iop}), $\vp=(\vp', (\oracle{\phi(g_i)})_{i\in [n]})$ and $\pp=(\pp', (\phi(g_i))_{i\in [n]})$ for some $\vp', \pp'$. Then  $\indexer_{\R'}^\lift$ outputs $\vp^\lift = (\vp', (\oracle{g_i})_{i\in [n]})$ and $\pp^\lift = (\pp', (g_i)_{i\in [n]})$.
        \vspace{0.3cm}
        
        %\end{algorithmic}%\newline
        	\textbf{Input:} Let $(\idx,\inp)$ be a well-formed index-instance pair for $\phi(\REL_{\gparams,\R, \cQ})$. $\prover_{\R'}^\lift$ and $\verifier_{\R'}^\lift$ receive $(\pp^\lift, \inp,\wit)$   and $(\vp^\lift, \inp)$ as input, respectively, where $(\vp^\lift, \pp^\lift) \gets \indexer_{\R'}^\lift(\gparams,\idx)$. Let  $(\vp, \pp) \gets \indexer_{\R'}(\gparams,\phi(\idx))$.\vspace{0.3cm}%\begin{algorithmic}[H]
        %{}
        	%\end{algorithmic} %\newline

        
        %\newline
	 %Additionally, $\prover_{\R'}^\lift$ receives $\wit$ such that
          %$(\idx, \inp; \wit)\in \phi(\REL_{\gparams,\R,  \cQ})$. 
          \textbf{Interactive phase:}   Let $k$ be the number of communication rounds  in $\prot_{\R'}$, and let $\cM_1, \ldots, \cM_{k+1}, \cC_1,\ldots, \cC_k$ be the message and challenge spaces of $\prot_{\R'}$.
        %\newline
        %
		\begin{algorithmic}[1]
			%\small
			\STATE Let $m_1$ be a first message output by $\prover_{\R'}(\pp, \phi(\inp);\phi(\wit))$. Then $\prover_{\R'}^{\lift}$ sends $m_1$ to $\verifier_{\R'}^{\lift}$.
            \STATE For $i=1,\ldots, k$,
            \begin{itemize}
                \item $\verifier_{\R'}^{\lift}$ uniformly samples a challenge $\rho_i$ in the challenge space $\cC_i$, and sends $\rho_i$ to $\prover_{\R'}^{\lift}$. 
                \item Let $m_{i+1}$ be output by $\prover_{\R'}(\pp,\phi(\inp);\phi(\wit))$ after having output messages $m_1, \ldots, m_{i}$ and received challenges $\rho_{1},\ldots, \rho_{i}$.  Then $\prover_{\R'}^{\lift}$ sends $m_{i+1}$ to $\verifier_{\R'}^{\lift}$. 
            \end{itemize}
            \STATE $\verifier_{\R'}^\lift$ outputs $\verifier_{\R'}(\vp, \phi(\inp), m_1, \rho_1,\ldots, m_k, \rho_{k}, m_{k+1})$. %: whenever $\verifier_{\R'}$ would query an oracle sent by $\prover_\R'$ during the interactive phase, $\verifier_{\R'}^\lift$ queries the same oracle at the same position. Whenever $\verifier_{\R'}$ would query an oracle $\phi(\oracle{h})$ in $\phi(\idx)$ or in $ \phi(\inp)$, $\verifier_{\R'}^\lift$ queries $\oracle{h}$ at the same position, and applies $\phi$ to the received value. Finally, $\verifier_{\R'}^\lift$ accepts if and only if $\verifier_{\R'}$ would accept when given the input $(\phi(\idx), \phi(\inp))$  the challenges sent by $\verifier_{\R'}^\lift$, and the values queried by $\verifier_{\R'}^\lift$.
            %For $i=1,\ldots, k$, let $\rho_i$ be the last challenge sent by $\verifier_{\R'}$
		\end{algorithmic}
	\end{algorithm}
          \end{definition}
    

        
        


        %begin{lemma}
        %     \albert{Is this needed? } Let $(\idx,\inp)$ be a well-formed index-instance pair for $\phi(\REL_{\R, \cQ})$. Let $\tr=(\rho_1, m_1,\ldots, \rho_k, m_{k})$ be a transcript of an execution of $\langle \prover_{\R'}^{\lift}(\idx, \inp), \verifier_{\R'}^{\lift}(\idx, \inp) \rangle$. Then $\tr=(\rho_1, m_1,\ldots, \rho_k, m_{k})$ is also a transcript of an execution of $\langle \prover_{\R'}(\phi(\idx), \phi(\inp)), \verifier_{\R'}^{\lift}(\phi(\idx), \phi(\inp)) \rangle$, and    
       % $\verifier_{\R'}^{\lift}(\idx, \inp, \tr)=1$ if and only if $\verifier_{\R'}(\phi(\idx), \phi(\inp), \tr)=1$.
       % \end{lemma}
       % \begin{proof}
  
       % \end{proof}

        %\albert{remove} Under the same setting as in the above \cref{d: Lift of an Algebraic PIOP}, given a malicious prover $\prover_{\R'}^{\lift,*}$ for $\prot_{\R'}^\lift$, we define a malicious prover $\prover_{\R'}^{*}$ for $\prot_{\R'}$. %similarly as we defined $\prover_{\R'}^{\lift}$. 
        %Namely, given $(\idx,\inp)$  as input, $\prover_{\R'}^{*}$  executes $\prover_{\R'}^{*}$ with input $(\phi(\idx),\phi(\inp))$ and sends to $\verifier_{\R'}^{\lift}$ the same oracles that  $\prover_{\R'}^{*}$ would send to  $\verifier_{\R'}$
    
        \begin{lemma}\label{l: lift_preserves_soundness}
        Let $\REL_{\gparams,\R,\cQ}$ be an algebraic indexed relation with parameters $(\gparams,\R,\cQ)$. Let $\phi:\R \to \R'$ be an efficient ring homomorphism. Let $\prot_{\R'}=(\indexer_{\R'}, \prover_{\R'}, \verifier_{\R'})$ be a  PIOP over $\R'$ for an algebraic indexed relation $\REL_{\gparams,\R',\phi(\cQ)}$. Suppose $\prot_{\R'}$ has soundness error $\eps_{\soundness}$ and completeness error $\eps_{\completeness}$. Then  $\prot_{\R'}^\lift$ is a PIOP over $\R$ for the relation $\phi(\REL_{\gparams,\R', \cQ})$ with soundness and completeness errors $\eps_{\soundness}^\lift, \eps_{\completeness}^\lift$ satisfying
        
            $$\eps_{\soundness}^\lift(\gparams, \idx,\inp) = \eps_{\soundness}(\gparams, \phi(\idx), \phi(\inp)), \quad \eps_{\completeness}^\lift(\gparams, \idx,\inp) = \eps_{\completeness}(\gparams, \phi(\idx), \phi(\inp))$$
        
        for all well-formed index-instance pairs $(\idx,\inp)$ for $\phi(\REL_{\gparams,\R, \cQ})$ (note that by \cref{r: well_formedness_preservation}, $(\phi(\idx),\phi(\inp))$ is well-formed as well).
        %
%        Further, $\prot_{\R'}^\lift$ has the same efficiency measures as $\prot_{\R'}$ except for the total complexity, which must take into account $m$ evaluations of $\phi$, for $m$ as in the definition of $\REL_{\R,\cQ}$ and the verifier time which is increased by $q$ evaluations of $\phi$, where $q$ is the number of oracle queries of $\prot_{\R'}$. \albert{Re efficiency measures: Not exactly? Statement to be revised} \luca{How does it sound now?}. 
        \end{lemma}

        
    
       


\subsection{Example: SHA256 hashing + RSA signature verification}

%%%%%%%%%%%%%%%%%%%%%%%%%%%%%%%%%%%%%%%%%%%%%%%%%%%%%%%%%%%%%%%%%%%%%%%%%%%%%%%%
%%%%%%%%%%%%%%%%%%%%%%%%%%%%%%%%%%%%%%%%%%%%%%%%%%%%%%%%%%%%%%%%%%%%%%%%%%%%%%%%
%%%%%%%%%%%%%%%%%%%%%%%%%%%%%%%%%%%%%%%%%%%%%%%%%%%%%%%%%%%%%%%%%%%%%%%%%%%%%%%%
\section{Zip+: A polynomial commitment scheme for $(\QQ[X])[\vec Y]$}
\subsection{The protocol}

\newpage

\begin{figure}[H]
	\begin{framed}
		\begin{description}
			\item[$\commit(\gp, f)$:] $f\in\polyspace$, $\vec Y\in\FF^{2\mu}$ 
				\begin{itemize}
					\item Compute matrix $V^f$ of size $\size\times\size$ and coefficients in $\QQ[X]$ that represents $f$ the following way:
						$$\mathbf V^f=(f_{b_1, b_2}(X))_{b_1, b_2 \in \{0,1\}^\mu}\in \QQ[X]^{\size\times\size}$$
					\item For each $i\in[\size]$, compute
					 $\vectorUi=\encode(f_{i,b_2}(X))_{b_2\in\{0,1\}^\mu}\in\QQ[X]^n$ and matrix 
					 $\matrixU=(\vectorUi)_{i\in [\size]}\in\QQ[X]^{\size\times n}$
					\item Output $\com=(\oracleUi)_{i\in[\size]}$, where $\oracleUi$ denote oracles to $\vectorUi$
				\end{itemize}
			\item[$\open(\gp, \com, f, \matrixU)$:] \ \
				\begin{itemize}
					\item Parse $(\vectorUi)_{i\in[\size]}\gets\matrixU$ and $(\oracleUi)_{i\in[\size]}\gets\com$
					\item Check that $\com$ consists of oracles to $\vectorUi$ and that $\vectorUi$ is $\delta$-close to $\encode_\code(\matrixV^f_i)$ for all $i$. \Aru{how}
					\item Reject if at any moment reading $\hat{u}$, $f$, or $\com$ some coefficient is not in $\QQ[X]$ or is larger than $poly()$
				\end{itemize}		
		\item[$\evaluation$:] \
		\item[$\testingP:$]
				\begin{enumerate}
					\item $V$ sends $r_1, \ldots, r_{\size}\in[0, q_0-1]$ and $\mathbf \alpha=(\alpha_0, \ldots, \alpha_{Bdeg})\in[0, q_0-1]^{Bdeg}$
					\item $P$ computes and outputs 
					$$\mathbf{v}=\sum_{i\in[\size]}r_i\matrixV^f_i(\mathbf{\alpha})\in \QQ^{\size}$$
					\item $V$ randomly chooses $J\subset[n]$ with $|J|=\Theta(\delta)$ and for each $j\in J$
						\begin{itemize}
							\item If $\mathbf v_j$ is not an integer or $|\mathbf v_j|>...$, $V$ rejects
							\item Queries $\hat{u}_{1,j}(X), \ldots, \hat{u}_{\size, j}(X)\in \QQ[X]$
							\item Rejects if
							\item Computes $(\hat{u}_{i,j}(\mathbf \alpha))_{i\in[\size]}$
							\item $V$ checks whether $\encode(\mathbf v)_j=\sum_{i\in[\size]}r_i\hat{u}_{i,j}(\mathbf{\alpha})$
						\end{itemize}
				\end{enumerate}
		\item[$\evaluationP:$]				
				\begin{enumerate}
					\item $V$ sends $r_1, \ldots, r_{\size}\in[0, q_0-1]$ and $\mathbf \alpha=(\alpha_0, \ldots, \alpha_{Bdeg})\in[0, q_0-1]^{Bdeg}$
					\item $P$ computes and outputs 
					$$\mathbf{v}=\sum_{i\in[\size]}\phi(q_{1,r})\phi(\matrixV^f_i)\in \FF_q^{\size}$$
					\item $V$ randomly chooses $J\subset[n]$ with $|J|=\Theta(\delta)$ and for each $j\in J$
					\begin{itemize}
						\item If $\mathbf v_j$ is not an integer or $|\mathbf v_j|>...$, $V$ rejects
						\item $V$ checks whether $\encode(\mathbf v_{q, x-\theta})_j=\sum_{i\in[\size]}\phi(q_{1,i})\phi_q(\hat{u}_{i,j}(X))$ and $\phi(y)=\sum_{i\in[\size]}(\mathbf v_{q, x-\theta})_i\phi(q_{2,i})$
					\end{itemize}
				\end{enumerate}
		\end{description}		
	\end{framed}
\end{figure}

	
%%%%%%%%%%%%%%%%%%%%%%%%%%%%%%%%%%%%%%%%%%%%%%%%%%%%%%%%%%%%%%%%%%%%%%%%%%%%%%%%
%%%%%%%%%%%%%%%%%%%%%%%%%%%%%%%%%%%%%%%%%%%%%%%%%%%%%%%%%%%%%%%%%%%%%%%%%%%%%%%%
\subsection{Proximity gaps for infinite rings}
\label{sec:proximity-gap}

\parhead{Informal proximity gap for proposed split encoding}
Consider a projectable linear code $C$ over $\integers$ with a systematic generator matrix $G = [I_{\dim} \ G']$, and a good random choice of $q$. The honest prover will encode a message $m \in \integers^{\dim}$ as $m \| (G'm \bmod q) \in \integers^{\dim} \times \integers_{q}^{\blocklength-\dim}$.

Now consider an arbitrary set of vectors over $v_1,\dotsc,v_{m} \in \integers^{\dim} \times \integers_{q}^{\blocklength-\dim}$, and let $(u_1,\dotsc,u_m) \equiv (v_1,\dotsc,v_{m}) \bmod q$. We know from~\cite[Theorem 4.1]{BKS18} that if there exists $u^\ast \in \set{\sum_{i=1}^{m} \alpha_i u_i \mid \alpha_1,\dotsc,\alpha_m \in \integers_q}$ such that $\reldist{u^\ast}{C} > \delta$, then for any $\epsilon > 0$ such that $\delta-\epsilon < \lambda/3$ it holds that
\[
\Pr_{\alpha \gets \integers_{q}^m} \left[\reldist{\sum_{i=1}^{m} \alpha_i u_i}{C} < \delta - \epsilon\right] < \frac{1}{\epsilon q}
\]
Since for any $\alpha$ we have $\reldist{\sum_{i=1}^{m} \alpha_i v_i}{C} \leq \reldist{\sum_{i=1}^{m} \alpha_i u_i}{C}$, this above result also holds for with respect to the $v_i$.

% We first recall the unique decoding version of the correlated agreement result for linear codes from~\cite{BKS18}.
% \begin{lemma}[Correlated agreement over affine spaces~\cite{BKS18}]
% \label{lemma:correlated-agreement}

% Let $V \subseteq \field^\blocklength$ be a linear space over a finite field $\field$ with $\Delta(C) = \lambda$. Suppose $u^\ast \in \ring^b$ satisfies $\Delta(u^\ast,C) > \delta$, and fix arbitrary $u \in \ring^b$. Let $\challengeset \subseteq \ring$ be an exceptional set. For $\epsilon > 0$ satisfying $\delta - \epsilon < \lambda/3$. Let
% \[
% A = \{\alpha \in \mathcal{C} \mid \distance{u^\ast + \alpha u}{C} < \delta - \varepsilon\}
% \]
% If $|A| > 1/\varepsilon$ then there exist $v, v^\ast \in C$ such that
% \[
% \bigl|\set{i \in [n] \mid (u_i = v_i) \ \wedge \ (u^\ast_i = v^\ast_i)}\bigr| \geq (1 - \delta)\cdot n
% \]
% \end{lemma}

%%%%%%%%%%%%%%%%%%%%%%%%%%%%%%%%%%%%%%%%%%%%%%%%%%%%%%%%%%%%%%%%%%%%%%%%%%%%%%%%
%%%%%%%%%%%%%%%%%%%%%%%%%%%%%%%%%%%%%%%%%%%%%%%%%%%%%%%%%%%%%%%%%%%%%%%%%%%%%%%%
\renewcommand*{\bibfont}{\small}
\printbibliography
%%%%%%%%%%%%%%%%%%%%%%%%%%%%%%%%%%%%%%%%%%%%%%%%%%%%%%%%%%%%%%%%%%%%%%%%%%%%%%%%
%%%%%%%%%%%%%%%%%%%%%%%%%%%%%%%%%%%%%%%%%%%%%%%%%%%%%%%%%%%%%%%%%%%%%%%%%%%%%%%%
%%%%%%%%%%%%%%%%%%%%%%%%%%%%%%%%%%%%%%%%%%%%%%%%%%%%%%%%%%%%%%%%%%%%%%%%%%%%%%%%
\end{document}
%%%%%%%%%%%%%%%%%%%%%%%%%%%%%%%%%%%%%%%%%%%%%%%%%%%%%%%%%%%%%%%%%%%%%%%%%%%%%%%%
%%%%%%%%%%%%%%%%%%%%%%%%%%%%%%%%%%%%%%%%%%%%%%%%%%%%%%%%%%%%%%%%%%%%%%%%%%%%%%%%
%%%%%%%%%%%%%%%%%%%%%%%%%%%%%%%%%%%%%%%%%%%%%%%%%%%%%%%%%%%%%%%%%%%%%%%%%%%%%%%%